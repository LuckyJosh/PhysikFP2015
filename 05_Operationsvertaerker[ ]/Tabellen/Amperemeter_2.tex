\begin{table}[!h]
	\centering
	\begin{tabular}{cccc}
		\toprule
		Frequenz & Strom & Eingangswiderstad & Leerlaufverstärkung\\
		$f$/\si{\kilo\hertz} & $I$/\si{\ampere} & $r_e$/\si{\ohm} & $V$\\
\midrule
		\num{100(1)} & \num{0.00209(2)} & \num{11(1)} & \num{908(80)}\\
		\num{200(2)} & \num{0.00207(2)} & \num{11(1)} & \num{899(79)}\\
		\num{500(5)} & \num{0.00207(2)} & \num{14(1)} & \num{690(47)}\\
		\num{750(8)} & \num{0.00207(2)} & \num{16(1)} & \num{591(35)}\\
		\num{1000(10)} & \num{0.00207(2)} & \num{19(1)} & \num{517(27)}\\
		\num{1500(15)} & \num{0.00207(2)} & \num{24(1)} & \num{413(18)}\\
		\num{2000(20)} & \num{0.00207(2)} & \num{28(1)} & \num{345(13)}\\
		\num{2500(25)} & \num{0.00207(2)} & \num{36(1)} & \num{276(8)}\\
		\num{3000(30)} & \num{0.00207(2)} & \num{41(1)} & \num{243(7)}\\
		\num{3500(35)} & \num{0.00207(2)} & \num{45(1)} & \num{217(6)}\\
		\num{4000(40)} & \num{0.00207(2)} & \num{50(1)} & \num{197(5)}\\
		\num{4500(45)} & \num{0.00207(2)} & \num{53(1)} & \num{188(4)}\\
		\num{5000(50)} & \num{0.00207(2)} & \num{60(1)} & \num{165(4)}\\
		\num{5500(55)} & \num{0.00207(2)} & \num{67(1)} & \num{147(3)}\\
		\num{6000(60)} & \num{0.00207(2)} & \num{72(1)} & \num{138(3)}\\
		\num{6500(65)} & \num{0.00207(2)} & \num{77(1)} & \num{129(2)}\\
		\num{7000(70)} & \num{0.00207(2)} & \num{82(1)} & \num{121(2)}\\
		\num{7500(75)} & \num{0.00207(2)} & \num{89(1)} & \num{111(2)}\\
		\num{8500(85)} & \num{0.00207(2)} & \num{94(1)} & \num{106(2)}\\
		\num{9000(90)} & \num{0.00207(2)} & \num{96(1)} & \num{103(2)}\\
		\num{10000(100)} & \num{0.00207(2)} & \num{103(1)} & \num{96(2)}\\
		\bottomrule
	\end{tabular}
	\caption{ Aus den gemessenen Spannungen der Amperemeterschaltung berechnete Werte des Stroms und des Eingangswiderstands
sowie die aus den letzteren berechneten Werte der Leerlaufverstärkung. \label{tab:amperemeter_2}}
\end{table}
