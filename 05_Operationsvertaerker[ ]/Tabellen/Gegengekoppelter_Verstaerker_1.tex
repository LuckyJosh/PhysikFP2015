\begin{table}[!h]
	\centering
	\begin{tabular}{ccc}
		\toprule
		Frequenz & Ausgangsspannung & Verstärkung\\
		$f$/\si{\kilo\hertz} & $U_{\mathrm{A}}$/\si{\milli\volt} & $V$\\
\midrule
		\num{1.00(1)} & \num{78(5)} & \num{1.11(7)}\\
		\num{5.00(5)} & \num{72(5)} & \num{1.03(7)}\\
		\num{10.0(1)} & \num{57(5)} & \num{0.81(7)}\\
		\num{15.0(1)} & \num{45(5)} & \num{0.64(7)}\\
		\num{20.0(2)} & \num{37(5)} & \num{0.53(7)}\\
		\num{25.0(2)} & \num{31(5)} & \num{0.44(7)}\\
		\num{30.0(3)} & \num{25(5)} & \num{0.36(7)}\\
		\num{35.0(4)} & \num{25(5)} & \num{0.36(7)}\\
		\num{75.0(8)} & \num{20(5)} & \num{0.29(7)}\\
		\num{100(1)} & \num{15(5)} & \num{0.21(7)}\\
		\bottomrule
	\end{tabular}
	\caption{Messwerte der Frequenz und der Ausgangsspannung der ersten Schaltung eines gegengekoppelten Verstärkers.
            Zusätzlich ist die Verstärkung dieser Schaltung angegeben. \label{tab:gegengekoppelter_verstaerker_1}}
\end{table}
