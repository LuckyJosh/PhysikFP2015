\begin{table}[!h]
	\centering
	\begin{tabular}{ccc}
		\toprule
		Widerstand & Widerstand & Verhältnis\\
		$R_{\mathrm{N}}$/\si{\kilo\ohm} & $R_1$/\si{\kilo\ohm} & $\frac{R_{\mathrm{N}}}{R_1}$\\
\midrule
		\num{100(1)} & \num{100(1)} & \num{1.00(1)}\\
		\num{10.0(1)} & \num{100(1)} & \num{0.100(1)}\\
		\num{10.0(1)} & \num{32.0(3)} & \num{0.312(4)}\\
		\num{32.0(3)} & \num{10.0(1)} & \num{3.20(5)}\\
		\bottomrule
	\end{tabular}
	\caption{Werte der 4 Widerstandspaare, die für die unterschiedlichen Schaltungen des gegengekoppelten Verstärkers 
verwendet wurden. Zusätzlich angegeben ist das Verhältnis dieser beiden Widerstände. \label{tab:gegengekoppelt_widerstaende}}
\end{table}
