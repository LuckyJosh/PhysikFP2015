% !TeX root = Protokoll.tex
\subsection{Gegengekoppelter Verstärker}

Für die vier Schaltungen eines gegengekoppelten Verstärkers, die 
in diesem Versuchsteil aufgebaut wurden, wurden die Widerstandspaare 
$R_{N}$ und  $R_{1}$ in \cref{tab:gegengekoppelt_widerstaende} verwendet.
Die unterschiedlichen Schaltungen werden der Reihenfolge in dieser Tabelle nach
(von oben nach unten) im Folgenden als erste bis vierte Schaltung bezeichnet.
Neben den jeweiligen Werten der Widerstände ist auch das Verhältnis 
$\sfrac{R_N}{R_1}$
angegeben welches nach \cref{eq:gegengekoppelt_verstaerkung} den Betrag der Verstärkung
der Schaltung angibt.

\begin{table}[!h]
	\centering
	\begin{tabular}{ccc}
		\toprule
		Widerstand & Widerstand & Verhältnis\\
		$R_N$/\si{\kilo\ohm} & $R_1$/\si{\kilo\ohm} & $\frac{R_N}{R_1}$\\
\midrule
		\num{100(1)} & \num{100(1)} & \num{1.00(1)}\\
		\num{10.0(1)} & \num{100(1)} & \num{0.100(1)}\\
		\num{10.0(1)} & \num{32.0(3)} & \num{0.312(4)}\\
		\num{32.0(3)} & \num{10.0(1)} & \num{3.20(5)}\\
		\bottomrule
	\end{tabular}
	\caption{Werte der 4 Widerstandspaare, die für die unterschiedlichen Schaltungen des gegengekoppelten Verstärkers 
verwendet wurden. Zusätzlich angegeben ist das Verhältnis dieser beiden Widerstände. \label{tab:gegengekoppelt_widerstaende}}
\end{table}


In den folgenden Tabellen \ref{tab:gegengekoppelter_verstaerker_1} -- \ref{tab:gegengekoppelter_verstaerker_4}
und den zugehörigen Abbildungen \ref{fig:gegengekoppelter_verstaerker_1} -- \ref{fig:gegengekoppelter_verstaerker_4}
ist jeweils die Verstärkung der vier Schaltungen in Abhängigkeit der Frequenz dargestellt.
Die Abbildungen erlauben den Vergleich der theoretischen Verstärkung, die dem 
Widerstandsverhältnis $\sfrac{R_N}{R_1}$ entspricht mit dem gemessenen Werte 
der 
mittleren maximalen Verstärkung. 

Es ist zu erkennen, dass für 
Widerstandsverhältnisse $ \geq 1$ der theoretische Wert gut mit dem gemittelten 
Wert übereinstimmt. Die relativen Abweichungen der Messwerte von den 
theoretischen Werten sind in einzeln in
\cref{tab:gegengekoppelter_verstaerker_max_verstaerkung} aufgeführt.
Bei den Widerstandsverhältnissen  $ < 1$ zeigt sich eine deutlich größere 
Abweichung der gemessenen Werte, welche aufgrund der doppellogarithmischen 
Darstellung jedoch größer zu seien scheint als diese in Wirklichkeit ist. 

Die Parameter der Ausgleichsrechnungen der Form
\begin{empheq}{equation}
	V^{\prime}(f) = f^{a} \cdot 10^{b}  
	\label{eq:fit}
\end{empheq}
der abfallenden Verstärkung sind zusammen mit der Grenzfrequenzen und dem 
Verstärkung-Bandbreiten-Produkten 
in \cref{tab:gegengekoppleter_verstaerker_parameter} eingetragen.
In doppellogarithmischer Darstellung haben die Parameter $a$ und $b$ die 
Bedeutung der Steigung respektive des $y$-Achsenabschnittes der Geraden.
Des Weiteren wurde auch die Leerlaufspannung $V$ mittels 
\cref{eq:leerlauf_verstaerkung} abgeschätzt und in 
\cref{tab:gegengekoppleter_verstaerker_parameter} eingetragen.

\begin{table}[!h]
	\centering
	\begin{tabular}{ccc}
		\toprule
		Frequenz & Ausgangsspannung & Verstärkung\\
		$f$/\si{\kilo\hertz} & $U_A$/\si{\milli\volt} & $V$\\
\midrule
		\num{1.00(1)} & \num{78(5)} & \num{1.11(7)}\\
		\num{5.00(5)} & \num{72(5)} & \num{1.03(7)}\\
		\num{10.0(1)} & \num{57(5)} & \num{0.81(7)}\\
		\num{15.0(1)} & \num{45(5)} & \num{0.64(7)}\\
		\num{20.0(2)} & \num{37(5)} & \num{0.53(7)}\\
		\num{25.0(2)} & \num{31(5)} & \num{0.44(7)}\\
		\num{30.0(3)} & \num{25(5)} & \num{0.36(7)}\\
		\num{35.0(4)} & \num{25(5)} & \num{0.36(7)}\\
		\num{75.0(8)} & \num{20(5)} & \num{0.29(7)}\\
		\num{100(1)} & \num{15(5)} & \num{0.21(7)}\\
		\bottomrule
	\end{tabular}
	\caption{Messwerte der Frequenz und der Ausgangsspannung der ersten Schaltung eines gegengekoppelten Verstärkers.
            Zusätzlich ist die Verstärkung dieser Schaltung angegeben. \label{tab:gegengekoppelter_verstaerker_1}}
\end{table}

\FloatBarrier
\begin{figure}[!h]
\centering
\includegraphics[scale=1]{../Grafiken/Gegengekoppelter_Verstaerker_1.pdf}
\caption{Doppellogarithmische Darstellung der Verstärkung der ersten gegengekoppelten Verstärkerschaltung in Abhängigkeit der Frequenz der Eingangsspannug. Zusätzlich wurden die Ausgleichsgerade
	durch die abfallenden Messwerte und eine senkrechte Gerade bei der Grenzfrequenz eingezeichnet. Ferner sind noch  zwei waagerechte Geraden dargestellt. Die eine markiert den Mittelwert der Messwerte im Plateau und die  andere
	den theoretischen Wert dieser Größe.
	\label{fig:gegengekoppelter_verstaerker_1}}
\end{figure}
\FloatBarrier

\begin{table}[!h]
	\centering
	\begin{tabular}{ccc}
		\toprule
		Frequenz & Ausgangsspannung & Verstärkung\\
		$f$/\si{\kilo\hertz} & $U_A$/\si{\milli\volt} & $V$\\
\midrule
		\num{0.100(1)} & \num{25(5)} & \num{0.36(7)}\\
		\num{1.00(1)} & \num{22(5)} & \num{0.31(7)}\\
		\num{5.00(5)} & \num{18(5)} & \num{0.26(7)}\\
		\num{10.0(1)} & \num{18(5)} & \num{0.26(7)}\\
		\num{20.0(2)} & \num{16(5)} & \num{0.23(7)}\\
		\num{50.0(5)} & \num{11(5)} & \num{0.16(7)}\\
		\bottomrule
	\end{tabular}
	\caption{Messwerte der Frequenz und der Ausgangsspannung der zweiten Schaltung eines gegengekoppelten Verstärkers.
            Zusätzlich ist die Verstärkung dieser Schaltung angegeben. \label{tab:gegengekoppelter_verstaerker_2}}
\end{table}

\FloatBarrier
\begin{figure}[!h]
\centering
\includegraphics[scale=1]{../Grafiken/Gegengekoppelter_Verstaerker_2.pdf}
\caption{\label{fig:gegengekoppelter_verstaerker_2}}
\end{figure}
\FloatBarrier

\begin{table}[!h]
	\centering
	\begin{tabular}{ccc}
		\toprule
		Frequenz & Ausgangsspannung & Verstärkung\\
		$f$/\si{\kilo\hertz} & $U_A$/\si{\milli\volt} & $V$\\
\midrule
		\num{0.0100(1)} & \num{37(5)} & \num{0.53(7)}\\
		\num{0.100(1)} & \num{35(5)} & \num{0.50(7)}\\
		\num{1.00(1)} & \num{38(5)} & \num{0.54(7)}\\
		\num{10.0(1)} & \num{32(5)} & \num{0.46(7)}\\
		\num{20.0(2)} & \num{30(5)} & \num{0.43(7)}\\
		\num{30.0(3)} & \num{25(5)} & \num{0.36(7)}\\
		\num{40.0(4)} & \num{19(5)} & \num{0.27(7)}\\
		\num{50.0(5)} & \num{19(5)} & \num{0.27(7)}\\
		\num{100(1)} & \num{16(5)} & \num{0.23(7)}\\
		\bottomrule
	\end{tabular}
	\caption{Messwerte der Frequenz und der Ausgangsspannung der dritten Schaltung eines gegengekoppelten Verstärkers.
            Zusätzlich ist die Verstärkung dieser Schaltung angegeben. \label{tab:gegengekoppelter_verstaerker_3}}
\end{table}

\FloatBarrier
\begin{figure}[!h]
\centering
\includegraphics[scale=1]{../Grafiken/Gegengekoppelter_Verstaerker_3.pdf}
\caption{\label{fig:gegengekoppelter_verstaerker_3}}
\end{figure}
\FloatBarrier

\begin{table}[!h]
	\centering
	\begin{tabular}{ccc}
		\toprule
		Frequenz & Ausgangsspannung & Verstärkung\\
		$f$/\si{\kilo\hertz} & $U_A$/\si{\milli\volt} & $V$\\
\midrule
		\num{0.100(1)} & \num{220(5)} & \num{3.14(8)}\\
		\num{1.00(1)} & \num{230(5)} & \num{3.29(9)}\\
		\num{5.00(5)} & \num{160(5)} & \num{2.29(8)}\\
		\num{10.0(1)} & \num{100(5)} & \num{1.43(7)}\\
		\num{20.0(2)} & \num{60(5)} & \num{0.86(7)}\\
		\num{30.0(3)} & \num{45(5)} & \num{0.64(7)}\\
		\num{50.0(5)} & \num{35(5)} & \num{0.50(7)}\\
		\num{100(1)} & \num{25(5)} & \num{0.36(7)}\\
		\bottomrule
	\end{tabular}
	\caption{Messwerte der Frequenz und der Ausgangsspannung der vierten Schaltung eines gegengekoppelten Verstärkers.
            Zusätzlich ist die Verstärkung dieser Schaltung angegeben. \label{tab:gegengekoppelter_verstaerker_4}}
\end{table}

\FloatBarrier
\begin{figure}[!h]
\centering
\includegraphics[scale=1]{../Grafiken/Gegengekoppelter_Verstaerker_4.pdf}
\caption{Doppellogarithmische Darstellung der Verstärkung der vierten gegengekoppelten Verstärkerschaltung in Abhängigkeit der Frequenz der Eingangsspannug. Zusätzlich wurden die Ausgleichsgerade
	durch die abfallenden Messwerte und eine senkrechte Gerade bei der Grenzfrequenz eingezeichnet. Ferner sind noch  zwei waagerechte Geraden dargestellt. Die eine markiert den Mittelwert der Messwerte im Plateau und die  andere
	den theoretischen Wert dieser Größe.\label{fig:gegengekoppelter_verstaerker_4}}
\end{figure}
\FloatBarrier

\begin{table}[!h]
	\centering
	\begin{tabular}{ccc}
		\toprule
		Theoretische Verstärkung & Gemessene Verstärkung & relative Abweichung\\
		$V^{\prime}$ & $\overline{V^{\prime}_{\mathrm{max}}}$ & $\Delta_{\mathrm{rel}} 
		V$/\si{\percent}\\
\midrule
		\num{1.000} & \num{1.071} & \num{7.143}\\
		\num{0.100} & \num{0.336} & \num{235.714}\\
		\num{0.312} & \num{0.514} & \num{64.571}\\
		\num{3.200} & \num{3.214} & \num{0.446}\\
		\bottomrule
	\end{tabular}
	\caption{ Vergleich der Werte der theoretischen Verstärkung der Schaltung, welche dem Verhältnis der verbauten 
Widerstände entspricht, mit dem Mittelwert der der gemessenen Verstärkungen vor dem linearen Abfall in doppellogarithmischer Darstellung. \label{tab:gegengekoppelter_verstaerker_max_verstaerkung}}
\end{table}

\begin{table}[!h]
	\centering
\begin{adjustbox}{width=1\textwidth}
	\begin{tabular}{ccccc}
		\toprule
		Steigung & $y$-Achsenabschnitt & Grenzfrequenz & Verstärkung-Bandbreite & Leerlaufverstärkung\\
		$a$ & $b$ & $f_{\mathrm{g}}$/\si{\kilo\hertz} & 
		$f_{\mathrm{g}}V^{\prime}_{\mathrm{max}}$/\si{\kilo\hertz} & $V$\\
\midrule
		\num{-0.61(4)} & \num{0.52(6)} & \num{11.025} & \num{11.813} & \num{-15.000}\\
		\num{-0.19(7)} & \num{-0.43(8)} & \num{10.562} & \num{3.546} & \num{-0.142}\\
		\num{-0.32(6)} & \num{0.00(8)} & \num{23.453} & \num{12.061} & \num{-0.796}\\
		\num{-0.68(2)} & \num{0.83(2)} & \num{5.013} & \num{16.112} & \num{-720.000}\\
		\bottomrule
	\end{tabular}
\end{adjustbox}
	\caption{ Parameter der Ausgleichskurven der abfallenden Verstärkung, sowie die Grenzfrequenz und das 
Verstärkung-Bandbreiten-Produkt für jeder der vier Schaltungen. Bezeichnet werden die Parameter mit Steigung und 
y-Achsenabschnitt, da die Parameter in doppellogarithmischer Darstellung diese Bedeutung haben. \label{tab:gegengekoppleter_verstaerker_parameter}}
\end{table}


Der Frequenzgang der Verstärkerschaltung lässt sich mit dem in 
\cref{fig:ersatzschaltbild} dargestellten Ersatzschaltbild eines idealen 
Operationsverstärkers 
mit nachgeschaltetem Tiefpass erklären.

\FloatBarrier
\begin{figure}[!h]
\centering
\includegraphics[scale=0.75]{../Grafiken/Ersatzschaltbild.jpg}
\caption{Ersatzschaltbild eines realen Operationsverstärkers 
	in dem dieser durch einen idealen Operationsverstärker und einen 
	nachgeschalteten Tiefpass ersetzt wird. Der Tiefpass beschreibt dabei 
	interne Widerstände und Kapazitäten des realen Operationsverstärkers.
	\label{fig:ersatzschaltbild}}
\end{figure}
\FloatBarrier

Die Messwerte des Frequenzgangs der Phasenbeziehung zwischen Ein- und 
Ausgangsspannung sind in \cref{tab:gegengekoppelter_verstaerker_phase}
aufgelistet und in \cref{fig:gegengekoppelter_verstaerker_phase} graphisch 
dargestellt. 

\begin{table}[!h]
	\centering
	\begin{tabular}{cc}
		\toprule
		Frequenz & Phase\\
		$f$/\si{\kilo\hertz} & $\Delta \phi$/\si{\degree}\\
\midrule
		\num{0.100(1)} & \num{175(5)}\\
		\num{0.200(2)} & \num{170(5)}\\
		\num{0.300(3)} & \num{168(5)}\\
		\num{0.400(4)} & \num{168(5)}\\
		\num{0.500(5)} & \num{165(5)}\\
		\num{1.00(1)} & \num{162(5)}\\
		\num{2.00(2)} & \num{150(5)}\\
		\num{3.00(3)} & \num{140(5)}\\
		\num{4.00(4)} & \num{130(5)}\\
		\num{5.00(5)} & \num{125(5)}\\
		\num{10.0(1)} & \num{110(5)}\\
		\num{20.0(2)} & \num{90(5)}\\
		\bottomrule
	\end{tabular}
	\caption{Messwerte der Frequenz und der Phase zwischen Eingangs- und Ausgangsspannung der vierten Schaltung 
eines gegengekoppelten Verstärkers. \label{tab:gegengekoppelter_verstaerker_phase}}
\end{table}

\FloatBarrier
\begin{figure}[!h]
\centering
\includegraphics[scale=1]{../Grafiken/Gegengekoppelter_Verstaerker_Phase.pdf}
\caption{\label{fig:gegengekoppelter_verstaerker_phase}}
\end{figure}
\FloatBarrier



\subsection{Amperemeter mit geringem Eingangswiderstand}

Die am Amperemeter aufgenommenen Messwerte sind in \cref{tab:amperemeter_1}
einzusehen, wobei zusätzlich die theoretische Ausgangsspannung 
$U_{A,\mathrm{theo}}$ nach \cref{eq:amperemeter_strom}
und die relative Abweichung der gemessenen Ausgangsspannung $U_A$ zu dieser 
bestimmt wurde. 
In  \cref{tab:amperemeter_2} sind die berechneten Werte für Strom $I$, 
Innenwiderstand $r_e$ und die Leerlaufverstärkung $V$ angegeben. 
Die graphischen Darstellungen der letzten beiden Größen in Abhängigkeit der
Frequenz sind in 
\cref{fig:amperemeter_eingangswiderstand} und 
\cref{fig:amperemeter_leerlaufverstaerkung} gezeigt.

\begin{table}[!h]
	\centering
	\begin{adjustbox}{width=1\textwidth}
	\begin{tabular}{cccccc}
		\toprule
		Frequenz & Generatorspannung & Eingangsspannung & Ausgangsspannung & Ausgangsspannung & Ausgangsspannung\\
		$f$/\si{\kilo\hertz} & $U_g$/\si{\milli\volt} & $U_E$/\si{\milli\volt} & $U_A$/\si{\volt} & $U_{A,\mathrm{theo}}$/\si{\volt} & $\Delta_{\mathrm{rel}}U_A$/\si{\percent}\\
\midrule
		\num{100(1)} & \num{0.209(1)} & \num{0.023(2)} & \num{20.0(1)} & \num{20.9(3)} & \num{4(2)}\\
		\num{200(2)} & \num{0.207(1)} & \num{0.023(2)} & \num{20.0(1)} & \num{20.7(3)} & \num{3(2)}\\
		\num{500(5)} & \num{0.207(1)} & \num{0.030(2)} & \num{20.0(1)} & \num{20.7(3)} & \num{3(2)}\\
		\num{750(8)} & \num{0.207(1)} & \num{0.035(2)} & \num{20.0(1)} & \num{20.7(3)} & \num{3(2)}\\
		\num{1000(10)} & \num{0.207(1)} & \num{0.040(2)} & \num{20.0(1)} & \num{20.7(3)} & \num{3(2)}\\
		\num{1500(15)} & \num{0.207(1)} & \num{0.050(2)} & \num{19.0(1)} & \num{20.7(3)} & \num{8(2)}\\
		\num{2000(20)} & \num{0.207(1)} & \num{0.060(2)} & \num{19.0(1)} & \num{20.7(3)} & \num{8(2)}\\
		\num{2500(25)} & \num{0.207(1)} & \num{0.075(2)} & \num{19.5(1)} & \num{20.7(3)} & \num{6(2)}\\
		\num{3000(30)} & \num{0.207(1)} & \num{0.085(2)} & \num{19.5(1)} & \num{20.7(3)} & \num{6(2)}\\
		\num{3500(35)} & \num{0.207(1)} & \num{0.095(2)} & \num{19.5(1)} & \num{20.7(3)} & \num{6(2)}\\
		\num{4000(40)} & \num{0.207(1)} & \num{0.105(2)} & \num{18.0(1)} & \num{20.7(3)} & \num{14(2)}\\
		\num{4500(45)} & \num{0.207(1)} & \num{0.110(2)} & \num{18.7(1)} & \num{20.7(3)} & \num{10(2)}\\
		\num{5000(50)} & \num{0.207(1)} & \num{0.125(2)} & \num{18.3(1)} & \num{20.7(3)} & \num{13(2)}\\
		\num{5500(55)} & \num{0.207(1)} & \num{0.140(2)} & \num{18.0(1)} & \num{20.7(3)} & \num{14(2)}\\
		\num{6000(60)} & \num{0.207(1)} & \num{0.150(2)} & \num{17.5(1)} & \num{20.7(3)} & \num{18(2)}\\
		\num{6500(65)} & \num{0.207(1)} & \num{0.160(2)} & \num{17.3(1)} & \num{20.7(3)} & \num{19(2)}\\
		\num{7000(70)} & \num{0.207(1)} & \num{0.170(2)} & \num{16.9(1)} & \num{20.7(3)} & \num{22(2)}\\
		\num{7500(75)} & \num{0.207(1)} & \num{0.185(2)} & \num{16.5(1)} & \num{20.7(3)} & \num{25(2)}\\
		\num{8500(85)} & \num{0.207(1)} & \num{0.195(2)} & \num{16.3(1)} & \num{20.7(3)} & \num{26(2)}\\
		\num{9000(90)} & \num{0.207(1)} & \num{0.200(2)} & \num{15.7(1)} & \num{20.7(3)} & \num{31(2)}\\
		\num{10000(100)} & \num{0.207(1)} & \num{0.215(2)} & \num{14.7(1)} & \num{20.7(3)} & \num{40(2)}\\
		\bottomrule
	\end{tabular}
\end{adjustbox}
	\caption{ Messwerte der Frequenz der Eingangsspannung, der Generator-, der Eingangs- und Ausgangsspannung
der Ampermeterschaltung. Zusätzlich ist die theoretische Ausgangsspannung und der realtive Unterschied zwischen dieser
und der gemessenen angegeben. \label{tab:amperemeter_1}}
	
\end{table}


\begin{table}[!h]
	\centering
	\begin{tabular}{cccc}
		\toprule
		Frequenz & Strom & Eingangswiderstad & Leerlaufverstärkung\\
		$f$/\si{\kilo\hertz} & $I$/\si{\ampere} & $r_e$/\si{\ohm} & $V$\\
\midrule
		\num{100(1)} & \num{0.00209(2)} & \num{11(1)} & \num{908(80)}\\
		\num{200(2)} & \num{0.00207(2)} & \num{11(1)} & \num{899(79)}\\
		\num{500(5)} & \num{0.00207(2)} & \num{14(1)} & \num{690(47)}\\
		\num{750(8)} & \num{0.00207(2)} & \num{16(1)} & \num{591(35)}\\
		\num{1000(10)} & \num{0.00207(2)} & \num{19(1)} & \num{517(27)}\\
		\num{1500(15)} & \num{0.00207(2)} & \num{24(1)} & \num{413(18)}\\
		\num{2000(20)} & \num{0.00207(2)} & \num{28(1)} & \num{345(13)}\\
		\num{2500(25)} & \num{0.00207(2)} & \num{36(1)} & \num{276(8)}\\
		\num{3000(30)} & \num{0.00207(2)} & \num{41(1)} & \num{243(7)}\\
		\num{3500(35)} & \num{0.00207(2)} & \num{45(1)} & \num{217(6)}\\
		\num{4000(40)} & \num{0.00207(2)} & \num{50(1)} & \num{197(5)}\\
		\num{4500(45)} & \num{0.00207(2)} & \num{53(1)} & \num{188(4)}\\
		\num{5000(50)} & \num{0.00207(2)} & \num{60(1)} & \num{165(4)}\\
		\num{5500(55)} & \num{0.00207(2)} & \num{67(1)} & \num{147(3)}\\
		\num{6000(60)} & \num{0.00207(2)} & \num{72(1)} & \num{138(3)}\\
		\num{6500(65)} & \num{0.00207(2)} & \num{77(1)} & \num{129(2)}\\
		\num{7000(70)} & \num{0.00207(2)} & \num{82(1)} & \num{121(2)}\\
		\num{7500(75)} & \num{0.00207(2)} & \num{89(1)} & \num{111(2)}\\
		\num{8500(85)} & \num{0.00207(2)} & \num{94(1)} & \num{106(2)}\\
		\num{9000(90)} & \num{0.00207(2)} & \num{96(1)} & \num{103(2)}\\
		\num{10000(100)} & \num{0.00207(2)} & \num{103(1)} & \num{96(2)}\\
		\bottomrule
	\end{tabular}
	\caption{ Aus den gemessenen Spannungen der Amperemeterschaltung berechnete Werte des Stroms und des Eingangswiderstands
sowie die aus den letzteren berechneten Werte der Leerlaufverstärkung. \label{tab:amperemeter_2}}
\end{table}


\FloatBarrier
\begin{figure}[!h]
\centering
\includegraphics[scale=1]{../Grafiken/Amperemeter_Eingangswiderstand.pdf}
\caption{Doppellogarithmische Darstellung des Eingangswiderstands der Amperemeterschaltung
	in Abhängigkeit der Frequenz. \label{fig:amperemeter_eingangswiderstand}}
\end{figure}
\FloatBarrier
\FloatBarrier
\begin{figure}[!h]
\centering
\includegraphics[scale=1]{../Grafiken/Amperemeter_Leerlaufverstaerkung.pdf}
\caption{Doppellogarithmische Darstellung des Leerlaufverstärkung der Amperemeterschaltung
	in Abhängigkeit der Frequenz.  \label{fig:amperemeter_leerlaufverstaerkung}}
\end{figure}
\FloatBarrier

\subsection{Integrator- und Differentiatorschaltung}

Die von Integrator- und Diffentiatorschaltung aufgenommenen Messwerte
befinden sich in \cref{tab:integrator_differentiator}.
Die doppellogaritmische Darstellung der Frequenzabhängigkeit der 
Ausgangsspannung des Integrators ist in \cref{fig:integrator_frequenz}
und die des Differentiators in \cref{fig:differentiator_frequenz} gezeigt.
Zusätzlich zu den Messwerte ist auf das Ergebnis einer Ausgleichsrechnung nach
\cref{eq:fit} abgebildet. Die Parameter der Ausgleichskurven sind:
\begin{empheq}{alignat=2}
	a_{\mathrm{int}} &= \num{-0.91(1)} \qquad& b_{\mathrm{diff}} = 
	\num{0.90(1)}\\
	a_{\mathrm{int}} &= \num{4.64(2)} \qquad& b_{\mathrm{diff}} = 
	\num{0.31(3)}
\end{empheq}
%integrator
%Steigung a           -0.906547929601287 +/- 0.01060464728719671
%y-Achsenabschnitt b  4.637839462233714 +/- 0.023588320273295587
%differentiator
%Steigung a           0.8994746172449467 +/- 0.01152732378072151
%y-Achsenabschnitt b  0.31231762902616494 +/- 0.0330777456740357

In den Abbildungen \ref{fig:integrator_oszilloskop_sinus} --  
\ref{fig:integrator_oszilloskop_rechteck} und 
\ref{fig:differentiator_oszilloskop_sinus} --  
\ref{fig:differentiator_oszilloskop_rechteck} sind jeweils die 
Ausgangsspannungen des Integrators respektive des Differentiators
für die drei Eingangsspannungen Sinus, Dreieck und Rechteck dargestellt.


\begin{table}[!h]
	\centering
	\begin{tabular}{ccc}
		\toprule
		Frequenz & Ausgangsspannung & Ausgangsspannung\\
		$f$/\si{\hertz} & $U_{A,\mathrm{int}}$/\si{\milli\volt} & $U_{A,\mathrm{diff}}$/\si{\milli\volt}\\
\midrule
		\num{100(1)} & \num{670(10)} & \num{140(10)}\\
		\num{200(2)} & \num{350(10)} & \num{240(10)}\\
		\num{300(3)} & \num{250(10)} & \num{350(10)}\\
		\num{400(4)} & \num{180(10)} & \num{450(10)}\\
		\num{500(5)} & \num{160(10)} & \num{550(10)}\\
		\num{600(6)} & \num{140(10)} & \num{640(10)}\\
		\num{700(7)} & \num{120(10)} & \num{740(10)}\\
		\num{800(8)} & \num{100(10)} & \num{840(10)}\\
		\num{900(9)} & \num{90(10)} & \num{920(10)}\\
		\num{1000(10)} & \num{80(10)} & \num{1040(10)}\\
		\bottomrule
	\end{tabular}
	\caption{ Messwerte der Frequenz der Eingangsspannung und der Ausgangsspannung für die Schaltungen eines Umkehrintegrators und -differentiators. \label{tab:integrator_differentiator}}
\end{table}


\FloatBarrier
\begin{figure}[!h]
\centering
\includegraphics[scale=1]{../Grafiken/Integrator_Frequenz.pdf}
\caption{\label{fig:integrator_frequenz}}
\end{figure}
\FloatBarrier
\FloatBarrier
\begin{figure}[!h]
\centering
\includegraphics[scale=1]{../Grafiken/Integrator_Oszilloskop_Sinus.pdf}
\caption{\label{fig:integrator_oszilloskop_sinus}}
\end{figure}
\FloatBarrier
\FloatBarrier
\begin{figure}[!h]
\centering
\includegraphics[scale=0.75]{../Grafiken/Integrator_Oszilloskop_Dreieck.pdf}
\caption{Vom Oszilloskop aufgenommene Ein- und Ausgangsspannungen der Integratorschaltung. Auf dem Eingang
	liegt hier eine Dreicksspannung. Die Ausgangsspannung in Form entspricht dem theoretisch
	zu erwartendem Verlauf (periodische Abfolge nach oben respektive nach unten geöffneter 
	Parabeln).\label{fig:integrator_oszilloskop_dreieck}}
\end{figure}
\FloatBarrier
\FloatBarrier
\begin{figure}[!h]
\centering
\includegraphics[scale=1]{../Grafiken/Integrator_Oszilloskop_Rechteck.pdf}
\caption{Vom Oszilloskop aufgenommene Ein- und Ausgangsspannungen der Integratorschaltung. Auf dem Eingang
	liegt hier eine Rechteckspannung.\label{fig:integrator_oszilloskop_rechteck}}
\end{figure}
\FloatBarrier
\FloatBarrier
\begin{figure}[!h]
\centering
\includegraphics[scale=1]{../Grafiken/Differentiator_Frequenz.pdf}
\caption{\label{fig:differentiator_frequenz}}
\end{figure}
\FloatBarrier
\FloatBarrier
\begin{figure}[!h]
\centering
\includegraphics[scale=1]{../Grafiken/Differentiator_Oszilloskop_Sinus.pdf}
\caption{\label{fig:differentiator_oszilloskop_sinus}}
\end{figure}
\FloatBarrier
\FloatBarrier
\begin{figure}[!h]
\centering
\includegraphics[scale=1]{../Grafiken/Differentiator_Oszilloskop_Dreieck.pdf}
\caption{\label{fig:differentiator_oszilloskop_dreieck}}
\end{figure}
\FloatBarrier
\FloatBarrier
\begin{figure}[!h]
\centering
\includegraphics[scale=0.75]{../Grafiken/Differentiator_Oszilloskop_Rechteck.pdf}
\caption{Vom Oszilloskop aufgenommene Ein- und Ausgangsspannungen der Differentiatorschaltung. Auf dem Eingang
	liegt hier eine Rechteckspannung.\label{fig:differentiator_oszilloskop_rechteck}}
\end{figure}
\FloatBarrier

\subsection{Schmitt-Trigger}

Die für den Aufbau der Schmitt-Trigger-Schaltung verwendeten Bauteile
und aufgenommenen Messwerte hatten die folgenden Größen:
\begin{empheq}{align*}
	R_1 &= \input{../Ergebnisse/schmitt_trigger_R1.txt}\\
	R_P &= \input{../Ergebnisse/schmitt_trigger_Rp.txt}\\
	U_K &= \input{../Ergebnisse/schmitt_trigger_UK.txt}\\
	U_A &= \input{../Ergebnisse/schmitt_trigger_UA.txt}\\
	U_B &= \frac{U_A}{2}
\end{empheq}
Die gemessene Kippspannung des Schmitt-Triggers $U_K$ weicht damit von dem 
nach \cref{eq:kippspannug} theoretisch bestimmten Wert $U_{K,\mathrm{theo}} = 
\SI{0.66(1)}{\volt}$ in etwa um \SI{68}{\percent} ab.


\subsection{Funktionsgenerator}

Für den Aufbau der Schaltung wurden die folgenden Bauteile verwendet:
\begin{empheq}{alignat*=2}
R &= \SI{560(6)}{\ohm}  \qquad& C = \SI{1.00(1)}{\micro\farad}\\
R_1 &= \SI{1000(10)}{\ohm}\qquad& R_0 = \SI{32.0(3)}{\kilo\ohm}\\
R_P &= \SI{100(1)}{\kilo\ohm}\qquad& R^{\prime}_0 = \SI{10.0(1)}{\kilo\ohm}
\end{empheq}
Die Schaltung wurde um die zwei Widerstände $R_0$ und $R^{\prime}_0$ ergänzt,
die sich so ergebene Schaltung ist in \cref{fig:funktionsgenerator_real} 
dargestellt.

\FloatBarrier
\begin{figure}[!h]
\centering
\includegraphics[scale=0.6]{../Grafiken/Funktionsgenerator_real.jpg}
\caption{Die verwendete Schaltung eines 
Funktionsgenerators. Es wurden zwei  zusätzliche Widerstände in Reihe 
(dargestellt ist nur die Summe beider Widerstände) geschaltet, um 
die Eingangspannung des zweiten Operationsverstärkers zu verringern.
\label{fig:funktionsgenerator_real}}
\end{figure}
\FloatBarrier

Die  Frequenz und Amplitude der Rechteck- respektive Dreiecksspannung 
ergaben sich zu:
\begin{empheq}{align*}
f_{\mathrm{R}} &= \SI{570(1)}{\hertz} \qquad& f_{\mathrm{D}} = 
\SI{570(1)}{\hertz}\\
U_{A,\mathrm{R}} &= \SI{0.235(1)}{\volt} \qquad& U_{A,\mathrm{D}} = 
\SI{0.165(1)}{\volt}
\end{empheq}
Die theoretische Frequenz ergibt sich nach \cref{eq:funktionsgenerator_frequenz}
mit der Korrektur durch die beiden Widerstände $R_0$ und $R^{\prime}_0$ zu
\begin{empheq}{equation}
	f_{\mathrm{\mathrm{R},\mathrm{theo}}} = 
	f_{\mathrm{\mathrm{D},\mathrm{theo}}} =  
	\SI{587(11)}{\hertz}.
\end{empheq}
Die Scheitelwerte der Spannungen ergeben sich ebenfalls nach Korrektur
durch $R_0$ und $R^{\prime}_0$ aus \cref{eq:funktionsgenerator_rechteck} 
respektive \cref{eq:funktionsgenerator_dreieck} zu
\begin{empheq}{align}
U_{\mathrm{A,\mathrm{R},\mathrm{theo}}} &= \SI{0.197(3)}{\volt}\\
U_{\mathrm{A,\mathrm{D},\mathrm{theo}}} &=  \SI{0.150(2)}{\volt}.
\end{empheq}

Die Ausgangsspannungen des Funktionsgenerators, die mit Hilfe des Oszilloskops
aufgenommen wurden sind in \cref{fig:funktionsgenerator} dargestellt.

\FloatBarrier
\begin{figure}[!h]
\centering
\includegraphics[scale=1]{../Grafiken/Funktionsgenerator.pdf}
\caption{\label{fig:funktionsgenerator}}
\end{figure}
\FloatBarrier



