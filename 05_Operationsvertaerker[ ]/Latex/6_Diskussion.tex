% !TeX root = Protokoll.tex
Im Folgenden sollen die während des Versuchs aufgenommenen Messwerte und durch 
die Auswertungen erhaltenen Ergebnisse noch einmal auf abschließend diskutiert 
und auf Plausibilität hin überprüft werden.

Das aufgenommene Frequenzverhalten der Verstärkung der gegengekoppelten Verstärkerschaltungen 
kann als plausibel angesehen werden. In allen vier Messreihen ist ein eindeutiges Einbrechen
der Verstärkung ab einer bestimmten Frequenz zu erkennen. Für die Schaltungen mit 
einen Widerstandsverhältnis $\geq 1$ weist der Mittelwert der gemessenen maximalen Verstärkung 
nur sehr geringe Abweichungen  (\SI{7}{\percent} respektive \SI{0.5}{\percent}) zur theoretischen 
Verstärkung auf. Bei den übrigen Schaltungen mit einem Widerstandsverhältnis~$<1$ weichen dies stark
ab. Dieses Verhalten ist durch ein gegeneinander Arbeiten der Leerlaufverstärkung und der Dämpfung
durch das gegeben Widerstandsverhältnis zu begründen.
Ein ähnliches Verhalten lässt sich bei den berechneten Verstärkung-Bandbreite-Produkten beobachten,
diese stimmen nicht miteinander überein (obwohl diese konstant sein sollten), liegen jedoch, mit Außnahme
einer Messung, in der selben Größenordnung.Die Abweichende Messung ist auch hier diejenige mit dem geringsten 
Widerstandsverhältnis und entsprechend der größten Dämpfung.\\
Aus den abgeschätzten Leerlaufverstärkungen lässt sich erkennen, dass die vierte Schaltung die
Bedingung $\sfrac{R_{\mathrm{N}}}{R_1} << V$ am besten erfüllt.
 
Die Amperemeterschaltung lieferte wie zu erwarten ein mit zunehmender Frequenz abnehmende Verstärkung 
und ein entsprechende Zunahme des Eingangswiderstands. Ferner stimmen die theoretisch  
berechneten Ausgangsspannungen mit zunächst geringen Fehlern $< \SI{10}{\percent}$ mit der gemessenen überein.
Für höhere Frequenzen lässt sich eine Zunahme dieser Abweichung beobachten. Der Grund für diese Beobachtung liegt in 
der mit zu nehmender Frequenz schlechter erfüllten Bedingung $R_{\mathrm{V}} >> r_{\mathrm{e}}$, welche
erfüllt sein muss, damit $r_{\mathrm{e}}$  nur von dem Strom und der Eingangsspannung abhängt.

Sowohl die Integrator- als auch die Differentiatorschaltung  liefern Ausgangsspannungen, die der Theorie
entsprechend anti-proportional respektive proportional zur Frequenz der Eingangsspannung sind. Wiederum 
beider Schaltungen erfüllen ihre Aufgaben vollständig im betrachteten Frequenzbereich.
Entsprechend sind die jeweiligen Ausgangsspannungen für die drei unterschiedlichen Eingangsspannungen
(Sinus, Rechteck, Dreieck) mit den zeitlich integrierten respektive differenzieren Verläufen 
dieser Spannungen zu identifizieren. Lediglich die Differenzierung der Rechteckspannung ist mit realen 
Schaltungen nur schwerlich zu bewerkstelligen, da die entsprechende Ausgangsspannung einem Delta-Kamm
entspräche, dessen unendliche Höhe und infinitesimale Breite der Peaks durch reale schaltungen 
nicht geliefert werden können. In \cref{fig:differentiator_oszilloskop_rechteck} ist jedoch im Ansatz 
hohe und schmale Peaks zu erkennen, die einem Delta-Kamm ähneln. Ferner lassen sich auch Effekte erkennen,
die durch die reale Elektronik erzeugt werden so zum Beispiel die Relaxation der Ausgangsspannung von einem 
Peak zur Null und die Auswirkung des Peaks in der Ausgangsspannung auf die Eingangspannung.

Die Funktionsweise des Schmitt-Triggers konnte in diesem Versuch qualitativ nachgewiesen werden,
ab einer bestimmten Eingangsspannung, der Kippspannung $U_{\mathrm{K}}$, konnte das Kippen der 
Ausgangsspannung $U_{\mathrm{A}}$ beobachtet werden. Die gemessene Kippspannung weicht dabei um 
etwa \SI{70}{\percent} von dem theoretischen Wert ab, liegt jedoch noch in der selben Größenordnung.
Grund der Abweichung könnte die Instabilität des Schaltungsverhaltens sein, die durch die Mitkopplung des
verbauten Operationsverstärkers hervorgerufen wird.

Durch Anwendung der notwendigen Korrekturen durch die beiden zusätzlichen Widerstände ergibt sich,
dass die gemessenen Frequenzen der erzeugten Spannungen des Funktionsgenerators mit einer ungefähren 
Abweichung von \SI{3}{\percent} dem theoretischen Wert entsprechen. Und auch die gemessenen Amplituden 
der Spannungen weisen Abweichungen von \SI{19}{\percent} (Rechteck) und \SI{10}{\percent} (Dreieck)
auf.

Zusammenfassend lässt sich sagen, dass trotz einiger Problem in der Durchführung des Versuch 
und im Umgang mit den verwendeten Schaltungen die erhaltenen Ergebnisse als plausibel zu bewerten sind.  



  


 