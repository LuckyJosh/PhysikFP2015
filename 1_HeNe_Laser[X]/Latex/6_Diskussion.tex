% !TeX root = Protokoll.tex
Im Folgenden werden die in der Auswertung erhaltenen Ergebnisse noch einmal abschließend zusammengefasst 
und auf ihre Plausibilität hin überprüft.\\

Die Auswertung der Messungen, der Intensitätsverteilung der beiden Moden in \cref{sec:TEM00} und \cref{sec:TEM10}
lieferte folgende Ergebnisse.\\
Für die Grundmode $\mathrm{TEM_{00}}$:\\
\begin{empheq}{align*}
I_0 &= \SI{31.78(4)}{\micro\ampere},\\
\omega &= \SI{1.453(2)}{\milli\meter}.
\end{empheq}

Und für die erste transversale Mode $\mathrm{TEM_{10}}$:
\begin{empheq}{align*}
I_0 &= \SI{1.75(2)}{\micro\ampere},\\
\omega &= \SI{-1.282(1)}{\milli\meter}.
\end{empheq}

Wie der Vergleich mit den Regressionskurven zeigt spiegeln die, in dieser Messung, aufgenommenen Werte die theoretische Intensitätsverteilung 
gut wider. 
Dieses trifft auch auf die Messung zur Polarisation in \cref{sec:Polarisation}
zu. Hier ergaben sich die folgenden Werte:
	\begin{empheq}{align*}
	I_0 &= \SI{15.62(4)}{\micro\ampere},\\
	\phi_0 &= \SI{2.430(2)}{rad},\\
	&= \SI{139.2(1)}{\degree}.
	\end{empheq}

Die aufgenommenen Messwerte werden auch hier gut durch die theoretische Verteilung beschreiben. In allen drei Messungen 
sind jedoch auch Abweichungen von der Theorie zu erkennen, diese sind vor allem durch die Empfindlichkeit der Photodiode 
zu erklären, da die Messung der Intensität stark von deren Ausrichtung abhing. Hinzu kommt vor allem für die 
Überprüfung der Stabilitätsbedingung in \cref{sec:Stabilitaetsbedingung}, dass nicht immer die höchstmögliche Intensität am Laser 
eingestellt werden konnte. \\
Die für die Prüfung der Stabilitätsbedingung aufgenommenen Messwerte folgen zwar dem durch die Theorie
beschriebenen Trend, jedoch zweigt die berechnete Ausgleichskurve der Form
	\begin{empheq}{equation*}
	g_1g_2(L) = a \cdot L^2 + b \cdot L + c,
	\end{empheq}
mit den Parametern,
	\addtocounter{equation}{-1}
	\begin{subequations}
		\begin{empheq}{align*}
		a &= \SI{-0.22(6)}{\per\square\meter},\\
		b &= \SI{-0.1(1)}{\per\meter},\\
		c &= \SI{0.33(4)}{}
		\end{empheq}
	\end{subequations}
erhebliche Abweichungen von der Theoriekurve.\\

Somit ergibt sich aus den Messwerten nur eine Bestätigung der Tatsache, dass der Laser im Bereich $L > \SI{1}{m}$ außerhalb seines
Stabilitätsbereiches ist.

Die Wellenlänge konnte als Ergebnis der entsprechenden Messreihe zu 
	\begin{empheq}{equation*}
	\mean{\lambda} = \SI[parse-numbers=false]{637 \pm 2(sys) \pm 2(stat)}{nm},
	\end{empheq} 
berechnet werden.
Die mit \eqref{eq:Fehler_relativ} berechnete relative Abweichung $\Delta_{\mathrm{rel}}\lambda = \SI{0.7}{\percent}$ vom Theoriewert $\lambda = 
\SI{632.8}{\nm}$~\cite{V61} zeigt eine sehr hohe Genauigkeit der Messung.

