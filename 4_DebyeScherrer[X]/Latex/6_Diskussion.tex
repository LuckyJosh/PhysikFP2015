% !TeX root = Protokoll.tex
Im Folgenden werden die in der Auswertung erhaltenen Ergebnisse noch einmal abschließend
diskutiert und auf ihre Plausibilität hin überprüft.\\

Für die untersuchte Probe 4 lieferte die Auswertung die folgenden Ergebnisse:
\begin{empheq}{align*}
	\text{Kristallstruktur:\quad} &\text{kubisch-flächenzentriert}\\ 
	\text{Gitterkonstate:\quad} &a_{\mathrm{Probe}} = \SI{3.620(8)}{\angstrom} 
\end{empheq} 
Dabei betrug die mittlere Abweichung der $ \tfrac{d_{\mathrm{bragg},1}}{d_{\mathrm{bragg},i}} $ Werte von den 
$\tfrac{m_i}{m_1}$ Werten
\begin{empheq}{equation}
	\Delta_{\sfrac{m_i}{m_1},\mathrm{Probe}} = \num{0.02}.
\end{empheq}
Diese geringe Abweichung und der in \cref{fig:gitterkonstante_probe} deutlich zu erkennende lineare 
Verlauf der Messwerte legen nahe, dass die Ergebnisse der Untersuchung von Probe 4 eine hohe Genauigkeit
aufweisen und plausibel sind.\\
Verglichen mit den Daten in \cite{PSECu} handelte es sich bei der untersuchten Probe 4 
um Kupfer, welches in fcc-Struktur mit einer Gitterkonstante von  $a_{\mathrm{Cu}} = \SI{3.615}{\angstrom} $ vorliegt.

Die Untersuchung des Salzes 1 lieferte die Ergebnisse:
\begin{empheq}{align*}
\text{Kristallstruktur:\quad} &\text{Cäsiumchlorid}\\ 
\text{Gitterkonstate:\quad} &a_{\mathrm{Salz}} = \SI{3.1(1)}{\angstrom} 
\end{empheq} 
Die Abweichung der $ \tfrac{d_{\mathrm{bragg},1}}{d_{\mathrm{bragg},i}} $ von den $\tfrac{m_i}{m_1}$
ist in diesem Fall mit 
\begin{empheq}{equation}
\Delta_{\sfrac{m_i}{m_1},\mathrm{Salz}} = \num{0.06},
\end{empheq}
jedoch größer als bei der untersuchten Probe. Und auch der in \cref{fig:gitterkonstante_salz} dargestellte 
lineare Zusammenhang zwischen der Gitterkontanten und dem Beugungswinkel ist weniger deutlich zu erkennen als 
noch bei den Werten der Probe. Diese größere Ungenauigkeit kann durch die starke Schwärzung des Films 
aufgrund der langen Belichtungszeit begründet sein. Diese gestaltete das genaue Ablesen der Positionen der
Beugungsreflexe schwierig. 
Für das Salz konnte keine passende Verbindung gefunden werden,
verglichen mit \cite{Ganesan86} liegt die Gitterkonstante von Cäsiumchlorid selbst bei $a_{\mathrm{CsCl}} = \SI{4.126}{\angstrom}$ und ist damit wesentlich größer als der berechnete Wert.
