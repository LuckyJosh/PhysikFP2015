\begin{table}[!h]
	\centering
	\begin{tabular}{ccccc}
		\toprule
		Winkel & Winkelfunktion & Bragg-Abstand & rel. Bragg-Abstand & norm. Faktor (fcc)\\
		$\theta$/\si{rad} & $\sin(\theta)$ & $d_{\mathrm{bragg}}$/\si{\angstrom} & $\frac{d_{\mathrm{bragg},1}}{d_{\mathrm{bragg},i}}$ & $\frac{m}{m_1}$\\
\midrule
		\num{0.384(7)} & \num{0.375(6)} & \num{2.06(3)} & \num{1.0(0)} & \num{1.000}\\
		\num{0.445(7)} & \num{0.431(6)} & \num{1.79(2)} & \num{1.15(2)} & \num{1.155}\\
		\num{0.650(7)} & \num{0.605(6)} & \num{1.27(1)} & \num{1.62(3)} & \num{1.633}\\
		\num{0.785(8)} & \num{0.707(5)} & \num{1.090(8)} & \num{1.89(3)} & \num{1.915}\\
		\num{0.838(7)} & \num{0.743(5)} & \num{1.037(7)} & \num{1.98(4)} & \num{2.000}\\
		\num{1.025(7)} & \num{0.855(4)} & \num{0.902(4)} & \num{2.28(4)} & \num{2.309}\\
		\num{1.196(7)} & \num{0.930(2)} & \num{0.828(2)} & \num{2.48(4)} & \num{2.517}\\
		\num{1.270(6)} & \num{0.955(2)} & \num{0.807(2)} & \num{2.55(4)} & \num{2.582}\\
		\bottomrule
	\end{tabular}
	\caption{Aus der Bragg-Bedingung berechneter Netzebenenabstand, 
                    sowohl absolut als auch in Relation zum Ersten angegeben. 
                     \label{tab:probe_braggabstand}}
\end{table}
