\begin{table}[!h]
	\centering
	\begin{tabular}{cccccc}
		\toprule
		Abstand & Abstand & Durchmesser & Winkel & Winkel & Winkel\\
		$x_1$/\si{mm} & $x_2$/\si{mm} & $D$/\si{mm} & $4\phi$/\si{rad} & $4\theta$/\si{rad} & $\theta$/\si{rad}\\
\midrule
		\num{41(1)} & \num{132(1)} & \num{91(1)} & \num{1.59(3)} & \num{4.69(3)} & \num{1.174(7)}\\
		\num{36(1)} & \num{137(1)} & \num{101(1)} & \num{1.76(3)} & \num{4.52(3)} & \num{1.130(7)}\\
		\num{31(1)} & \num{143(1)} & \num{112(1)} & \num{1.95(3)} & \num{4.33(3)} & \num{1.082(7)}\\
		\num{21(1)} & \num{152(1)} & \num{131(1)} & \num{2.29(3)} & \num{4.00(3)} & \num{0.999(7)}\\
		\num{17(1)} & \num{156(1)} & \num{139(1)} & \num{2.43(3)} & \num{3.86(3)} & \num{0.964(7)}\\
		\num{13(1)} & \num{160(1)} & \num{147(1)} & \num{2.57(3)} & \num{3.72(3)} & \num{0.929(7)}\\
		\num{5(1)} & \num{177(1)} & \num{172(1)} & \num{3.00(3)} & \num{3.28(3)} & \num{0.820(7)}\\
		\bottomrule
	\end{tabular}
	\caption{Gemessener Abstand der Beugungsreflexe des untersuchten Salzes von dem linken Rand (Seite des Strahleintritts) des 
                    Filmstreifens, die aus diesen berechneten Durchmesser der Beugungsringe und daraus folgenden 
                    Beugungswinkeln.  
                     \label{tab:salz_links}}
\end{table}
