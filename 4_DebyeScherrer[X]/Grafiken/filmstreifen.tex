\begin{figure}[!h]
	\centering
	\begin{adjustbox}{width=12cm,height=3.5cm}
		

	\begin{tikzpicture}
	\draw (0.0,0.0)--(12.0,0.0)--(12.0,3.0) -- (0.0,3.0) --(0.0,0.0);
	\draw (3.0,1.5) circle (3mm) node {E};
	\draw (9.0,1.5) circle (3mm) node {A};
	
	\draw  ($(3.0,1.5) + (143.1:25mm)$) arc (143.1:216.9:25mm);
	\draw ($(3.0,1.5) + (-36.9:25mm)$) arc (-36.9:36.9:25mm);
	
	\draw  ($(9.0,1.5) + (139.3:23mm)$) arc (139.3:221.3:23mm);
	\draw  ($(9.0,1.5) + (-40.7:23mm)$) arc (-40.7:40.7:23mm);
	
	\draw[thin,dotted] (11.3,1.5)--(11.3,0);
	\draw[|-|,thick] (12.01,0)--(11.3,0) node[midway,below] {$x_1$} ;
	\draw[thin,dotted] (6.7,1.5)--(6.7,0);
	\draw[|-|,thick] (12.0,0)--(6.7,0) node[midway,below] {$x_2$} ;
	
	\draw[thin,dotted] (5.5,1.5)--(5.5,0);
	\draw[|-|,thick] (0.0,0.0)--(5.5,0) node[midway,below] {$x_2$} ;
	\draw[thin,dotted] (0.5,1.5)--(0.5,0);
	\draw[|-|,thick] (-0.01,0)--(0.5,0) node[midway,below] {$x_1$} ;
	\end{tikzpicture}
	\end{adjustbox}
	\caption{Schematische Darstellung des Filmstreifens mit Beugungsringen und der
		aufgenommenen Messwerte. E markiert den Strahleintritt und A den Strahlaustritt. \label{fig:messwerte}}	
\end{figure}