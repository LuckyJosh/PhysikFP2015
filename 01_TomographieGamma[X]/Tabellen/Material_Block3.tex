\begin{table}[!h]
	\centering
\begin{adjustbox}{width=\textwidth}
	\begin{tabular}{clcclc}
		\toprule
		Elementarwürfel & Material & rel. Abweichung & Elementarwürfel & Material & rel. Abweichung\\
		$i$ & & $\Delta_{rel}\mu/\%$ & $i$ & &$\Delta_{rel}\mu /\%$\\
\midrule
		\num{1} & Blei & \num{14}   & \num{6}& Blei& \num{14}\\
		\num{2} & Blei/Messing & \num{25}/\num{34}& \num{7}& Blei & \num{11}\\
		\num{3} & Messing  &\num{18} &\num{8}& Blei& \num{25}\\
		\num{4} & Messing & \num{30}  & \num{9}& Messing& \num{8}\\
		\num{5} & Messing/Blei & \num{28}/\num{28} &  &   &  \\
		
		\bottomrule
	\end{tabular}
\end{adjustbox}
	\caption{Zuordnung der möglichen Materialien zu den Einheitswürfeln des Würfels 3
		durch Vergleich der Absorptionskoeffizienten mit den für die Würfel 1 und 2 bestimmten 
		Werten. Liegt das Messergebnis zwischen diesen Werten so wurden beide möglichen Materialien angegeben, wobei das erste den geringeren Unterschied zum Messwert aufweist. \label{tab:Materialien_Block3}}
\end{table}
