\begin{table}[!h]
	\centering
\begin{adjustbox}{width=\textwidth}
	\begin{tabular}{cccc}
		\toprule
		Elementarwürfel & Abs. Koeffizient Würfel 1 & Abs. Koeffizient Würfel 2 & Abs. Koeffizient Würfel 3\\
		$i$ & $\mu_{1,i}$ & $\mu_{2,i}$ & $\mu_{3,i}$\\
\midrule
		\num{1} & \num{0.516(9)} & \num{0.95(1)} & \num{0.78(1)}\\
		\num{2} & \num{0.521(7)} & \num{0.880(8)} & \num{0.679(7)}\\
		\num{3} & \num{0.393(9)} & \num{0.76(1)} & \num{0.42(1)}\\
		\num{4} & \num{0.477(7)} & \num{0.961(8)} & \num{0.352(6)}\\
		\num{5} & \num{0.499(8)} & \num{0.87(1)} & \num{0.648(8)}\\
		\num{6} & \num{0.649(7)} & \num{1.056(8)} & \num{1.030(8)}\\
		\num{7} & \num{0.533(9)} & \num{0.84(1)} & \num{0.80(1)}\\
		\num{8} & \num{0.569(7)} & \num{1.147(8)} & \num{0.681(7)}\\
		\num{9} & \num{0.394(9)} & \num{0.65(1)} & \num{0.47(1)}\\
		\bottomrule
	\end{tabular}
\end{adjustbox}
	\caption{Unter Verwendug der Methode der kleinsten Quadrate bestimmte Absorbtionskoeffizeinten 
und deren Fehler der jeweils neun Elementarwürfel in den vermessenen drei Würfeln. \label{tab:Absorbtionskoeffizienten}}
\end{table}
