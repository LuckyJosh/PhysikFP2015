\begin{table}[!h]
	\centering
	\begin{tabular}{clcl}
		\toprule
		Elementarwürfel & Material & Elementarwürfel & Material\\
		$i$ & & $i$ & \\
\midrule
		\num{1} & Messing / Blei & \num{6}& Blei\\
		\num{2} & Messing & \num{7}& Messing / Blei \\
		\num{3} & Eisen  & \num{8}& Messing\\
		\num{4} & Aluminium / Eisen   & \num{9}& Eisen\\
		\num{5} & Messing & &       \\
		
		\bottomrule
	\end{tabular}
	\caption{Zuordnung der möglichen Materialien zu den Einheitswürfeln des Würfels 3
		durch Vergleich der Absorptionskoeffizienten mit den Literaturwerten. Liegt das Messergebnis 
		zwischen zwei Literaturwerten so wurden beide möglichen Materialien angegeben, wobei das erste den
		geringeren Unterschied zum Messwert aufweist. \label{tab:Materialien_Block3}}
\end{table}
