\begin{table}[!h]
	\centering
\begin{adjustbox}{width=\textwidth}
	\begin{tabular}{clcclc}
		\toprule
		Elementarwürfel & Material & rel. Abweichung & Elementarwürfel & Material & rel. Abweichung\\
		$i$ & & $\Delta_{rel}\mu/\%$ & $i$ & &$\Delta_{rel}\mu /\%$\\
\midrule
		\num{1} & Messing & \num{26}   & \num{6}& Blei& \num{17}\\
		\num{2} & Messing & \num{10}& \num{7}& Messing / Blei & \num{29}/\num{36}\\
		\num{3} & Eisen / Messing  &\num{29}/\num{33} &\num{8}& Messing & \num{10}\\
		\num{4} & Eisen/Messing & \num{39}/\num{43}  & \num{9}& Eisen& \num{20}\\
		\num{5} & Messing & \num{5} &  &   &  \\
		
		\bottomrule
	\end{tabular}
\end{adjustbox}
	\caption{Zuordnung der möglichen Materialien zu den Einheitswürfeln des Würfels 3
		durch Vergleich der Absorptionskoeffizienten mit den Literaturwerten. Liegt das Messergebnis 
		zwischen zwei Literaturwerten so wurden beide möglichen Materialien angegeben, wobei das erste den
		geringeren Unterschied zum Messwert aufweist. \label{tab:Materialien_Block3}}
\end{table}
