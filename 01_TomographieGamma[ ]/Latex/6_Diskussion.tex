% !TeX root = Protokoll.tex

Die Auswertung der Messwerte und der Vergleich der Ergebnisse mir den entsprechenden 
Literaturwerten zeigt, dass das in diesem Versuch durchgeführte Verfahren 
der Tomographie mit Gamma-Strahlung plausible Ergebnisse liefert. Die bestimmten 
Absorptionskoeffizienten für die Würfel 1 und 2 weisen die richtige Größenordnung 
auf. Die allgemein sehr großen relativen Abweichungen von den Literaturwerten lassen 
sich durch mögliche Ungenauigkeiten der Messung erklären. 
Zunächst können sowohl die in der Theorie vernachlässigte Breite des $\gamma$-Strahls
als auch eine fehlerhafte Positionierung der Würfel zu einer Abweichung von 
den Literaturwerten führen, da durch diese beiden Einflüsse die Längen in der Matrix $A$ 
nicht mehr den realen entsprechen.

Desweiteren werden durch die verwendete Messmethode die Absorptionskoeffizienten in Abhängigkeit 
von einander berechnet. Diese Abhängigkeit ist daran zu erkennen, dass die Elemente der 
berechneten Kovarianzmatrix $V[\vec{\mu}]$ aus \cref{eq:KleinsteQuadrateFehler}, die nicht auf der Hauptdiagonalen
liegen, die Kovarianzen, zwar kleiner als die Elemente der Hauptdiagonale sind, jedoch nicht verschwinden.
Diese Abhängigkeit zeigt sich auch bei Betrachtung der Ergebnisses für die Absorptionskoeffizienten in 
\cref{tab:Absorbtionskoeffizienten}. Hier zeigt sich, dass die Elementarwürfel 1 und 9, die jeweils
als letzter Würfel bei den diagonalen Projektionen 2 und 11 durchstrahlt werden einen merklich geringeren 
Absorptionskoeffizienten aufweisen als die übrigen Elementarwürfel. Die teilweise großen relativen 
Abweichungen  können somit auch durch diese Abhängigkeit der verschiedenen Absorptionskoeffizienten erklärt werden.
 