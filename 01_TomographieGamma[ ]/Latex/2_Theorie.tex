% !TeX root = Protokoll.tex
\subsection{Absorption}

\subsection{Tomographie}
Bei der Tomographie wird ein Objekt von mehreren Seiten bestrahlt und die Intensität der abgeschwächten Strahlung gemessen.
Daraus entsteht ein mehrdimensionale Darstellung des bestrahlten Objektes.
Dies resultiert daraus, dass die Strahlung in unterschiedlichen Materialien unterschiedlich stark abgeschwächt werden, weil sie nicht den selben Absorptionskoeffizienten $\mu$ besitzen.\\
\begin{wrapfigure}[16]{r}{0.5\textwidth}
	\centering
	\includegraphics[width=0.5\textwidth]{../Grafiken/Tikz/tikz-Projektionen.pdf}
	\caption{Die Darstellung der Verwendeten Projektionen.\label{fig:Projektion}}
\end{wrapfigure}
In diesem Versuch wird dazu ein Würfel benutzt, der aus $3\times3\times3$ Elementarwürfeln besteht.
Die Intensität die durch verschiedene Stoffen abgeschwächt wird lässt sich dabei schreiben als
\begin{align}
	I = I_0e^{-\sum \mu_i d_i},
\end{align}
darin ist $I_0$ die maximal Intensität, $\mu_i$ der Absorptionskoeffizient des Materials und $d_i$ die in dem Material zurückgelegte Strecke. 
Daraus kann für die unterschiedlichen Projektionen $j$ mit den Ausgangsintensitäten $I_j$ geschrieben werden als
\begin{align}
	\sum \mu_j d_j = -\ln\left(\frac{I_j}{I_0}\right)=:N_j
\end{align}
\newpage
Diese Gleichung kann mithilfe von \cref{fig:Projektion} in eine Matrixdarstellung umgeschrieben werden.

\begin{align}
&d \cdot
	\underbrace{
	\begin{pmatrix}
		0 & 0 & 0  & 0 & 0 & \sqrt{2} & 0  & \sqrt{2} & 0\\
		 0&0  &\sqrt{2} & 0 & \sqrt{2} & 0 & \sqrt{2} & 0 & 0\\
		 0& \sqrt{2} & 0 & \sqrt{2} & 0 &0 &0  &0  &0 \\
		0&0 &0 &0 &0 &0 & 1 & 1 & 1\\
		0&0 &0 & 1 & 1 & 1 &0 &0 &0 \\
		1 & 1& 1 & 0&0 &0 &0 &0 &0\\
		0&0 &0 & \sqrt{2} & 0 &0 &0& \sqrt{2} &0\\
		\sqrt{2} & 0 & 0& 0 &\sqrt{2} &0 &0 &0 &\sqrt{2}\\
		0 & \sqrt{2} & 0 & 0 & 0 & \sqrt{2} & 0 & 0 &0\\
		1 &0 &0  &1 &0 &0 & 1 & 0 &0 \\
		0 & 1 & 0 & 0 & 1 & 0 & 0 & 1 &0 \\
		0 & 0& 1 & 0 & 0 & 1 & 0 & 0 & 1
	\end{pmatrix}
	}_{A}
	\cdot
	\underbrace{
	\begin{pmatrix}
		\mu_1\\
		\mu_2\\
		\mu_3\\
		\mu_4\\
		\mu_5\\
		\mu_6\\
		\mu_7\\
		\mu_8\\
		\mu_9
	\end{pmatrix}
	}_{\vec{\mu}}
	=
	\underbrace{
	\begin{pmatrix}
		N_1\\
		N_2\\
		N_3\\
		N_4\\
		N_5\\
		N_6\\
		N_7\\
		N_8\\
		N_9		
	\end{pmatrix}
	}_{\vec{N}}\nonumber
\end{align}
\begin{align}	
	\Rightarrow A\cdot\vec{\mu}=\vec{N}
\end{align}
Weil $A$ eine $12\times 9 $ ist, wird daraus
\begin{align}
	\vec{\mu}=\left(A^TA\right)^{-1}\cdot A^T\cdot\vec{N}.
\end{align}
