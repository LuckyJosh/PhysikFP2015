% !TeX root = Protokoll.tex
Im Folgenden werden die in der Auswertung erhaltenen Ergebnisse
noch einmal zusammen gefasst und auf ihre Plausibilität hin überprüft.

Die linearen Regeressionen der magnetischen Flussdichte
$B$ in Abhängigkeit der Stromstärke $I$ lieferten die $y$-Achsenabschnitte
$b_{\mathrm{zu}} = \SI{30(10)}{\milli\tesla}$ und $b_{\mathrm{ab}} = \SI{37(8)}{\milli\tesla}$.
Diese Werte zeigen die aufgrund der Hysterese des Magneten zurückbleibende Magnetisierung,
da der Nulldurchgang bei abnehmendem Strom höher liegt. Da dieser Effekt zuerwarten ist,
sind die erhaltenen Ergebnisse als plausibel zu bewerten.

Die in \cref{tab:lande_ergebnis} dargestellten Ergebnisse für die zu bestimmenden
Landé-Faktoren der Übergänge, zeigen im  Vergleich mit den theoretisch Erwarteten
nur geringe Abweichungen von maximal \SI{10(4)}{\percent}. Damit sind auch die
Ergebnisse dieser Messung als plausibel anzusehen.
Ferner lässt sich aus diesen Ergebnissen schließen, dass der in diesem Versuch
verwendete Aufbau sehr gut für die Bestimmung kleinster Energie- respektive
Wellenlängenänderung geeignet ist.
