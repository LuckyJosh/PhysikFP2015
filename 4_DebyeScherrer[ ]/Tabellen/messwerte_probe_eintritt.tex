\begin{table}[!h]
	\centering
	\begin{tabular}{cccccc}
		\toprule
		Abstand & Abstand & Durchmesser & Winkel & Winkel & Winkel\\
		$x_1$/\si{mm} & $x_2$/\si{mm} & $D$/\si{mm} & $4\phi$/\si{rad} & $4\theta$/\si{rad} & $\theta$/\si{rad}\\
\midrule
		\num{3(1)} & \num{171(1)} & \num{168(1)} & \num{2.93(3)} & \num{3.35(3)} & \num{0.838(7)}\\
		\num{24(1)} & \num{149(1)} & \num{125(1)} & \num{2.18(3)} & \num{4.10(3)} & \num{1.025(7)}\\
		\num{43(1)} & \num{129(1)} & \num{86(1)} & \num{1.50(3)} & \num{4.78(3)} & \num{1.196(7)}\\
		\num{52(1)} & \num{121(1)} & \num{69(1)} & \num{1.20(3)} & \num{5.08(3)} & \num{1.270(6)}\\
		\bottomrule
	\end{tabular}
	\caption{Gemessener Abstand der Beugungsreflexe der untersuchten Probe von dem linken Rand (Seite des Strahleintritts) des 
                    Filmstreifens, die aus diesen berechneten Durchmesser der Beugungsringe und daraus folgenden 
                    Beugungswinkel.  
                     \label{tab:probe_links}}
\end{table}
