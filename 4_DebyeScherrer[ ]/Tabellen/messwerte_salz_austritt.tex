\begin{table}[!h]
	\centering
	\begin{tabular}{ccccc}
		\toprule
		Abstand & Abstand & Durchmesser & Winkel & Winkel\\
		$x_1$/\si{mm} & $x_2$/\si{mm} & $D$/\si{mm} & $4\theta$/\si{rad} & $\theta$/\si{rad}\\
\midrule
		\num{8(1)} & \num{165(1)} & \num{157(1)} & \num{2.74(3)} & \num{0.685(7)}\\
		\num{12(1)} & \num{161(1)} & \num{149(1)} & \num{2.60(3)} & \num{0.650(7)}\\
		\num{20(1)} & \num{152(1)} & \num{132(1)} & \num{2.30(3)} & \num{0.576(7)}\\
		\num{25(1)} & \num{148(1)} & \num{123(1)} & \num{2.15(3)} & \num{0.537(7)}\\
		\num{29(1)} & \num{143(1)} & \num{114(1)} & \num{1.99(3)} & \num{0.497(7)}\\
		\num{39(1)} & \num{133(1)} & \num{94(1)} & \num{1.64(3)} & \num{0.410(7)}\\
		\num{51(1)} & \num{121(1)} & \num{70(1)} & \num{1.22(3)} & \num{0.305(6)}\\
		\num{60(1)} & \num{113(1)} & \num{53(1)} & \num{0.93(3)} & \num{0.231(6)}\\
		\bottomrule
	\end{tabular}
	\caption{Gemessener Abstand der Beugungsreflexe des untersuchten Salzes von dem rechten Rand (Seite des Strahlaustritts) des 
                    Filmstreifens, die aus diesen berechneten Durchmesser der Beugungsringe und daraus folgenden 
                    Beugungswinkeln.  
                     \label{tab:salz_rechts}}
\end{table}
