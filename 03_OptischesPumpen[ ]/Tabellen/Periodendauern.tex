\begin{table}[!h]
	\centering
	\begin{adjustbox}{width=\textwidth}
	\begin{tabular}{ccccccc}
		\toprule
		Spannung & Zeitdifferenz & \# Perioden & Periodendauer & Zeitdifferenz & \# Perioden & Periodendauer\\
		$U$/\si{\volt} & $\Delta t_{1}$/\si{\second} & $N_{1}$ & $T_{1}$/\si{\second} & $\Delta t_{2}$/\si{\second} & $N_{2}$ & $T_{2}$/\si{\second}\\
\midrule
		\num{1} & \num{9.0(1)} & \num{3} & \num{3.00(3)} & \num{1.0(1)} & \num{1} & \num{1.0(1)}\\
		\num{2} & \num{6.4(1)} & \num{4} & \num{1.60(3)} & \num{2.5(1)} & \num{2} & \num{1.26(5)}\\
		\num{3} & \num{6.4(1)} & \num{6} & \num{1.07(2)} & \num{4.3(1)} & \num{4} & \num{1.07(3)}\\
		\num{4} & \num{5.0(1)} & \num{7} & \num{0.71(1)} & \num{2.0(1)} & \num{3} & \num{0.68(3)}\\
		\num{5} & \num{5.2(1)} & \num{8} & \num{0.65(1)} & \num{4.5(1)} & \num{7} & \num{0.65(1)}\\
		\num{6} & \num{3.2(1)} & \num{6} & \num{0.54(2)} & \num{4.4(1)} & \num{9} & \num{0.49(1)}\\
		\num{7} & \num{2.3(1)} & \num{8} & \num{0.28(1)} & \num{4.6(1)} & \num{12} & \num{0.387(8)}\\
		\num{8} & \num{2.2(1)} & \num{8} & \num{0.27(1)} & \num{4.6(1)} & \num{13} & \num{0.351(8)}\\
		\num{9} & \num{2.2(1)} & \num{8} & \num{0.27(1)} & \num{3.1(1)} & \num{10} & \num{0.31(1)}\\
		\num{10} & \num{2.1(1)} & \num{8} & \num{0.27(1)} & \num{3.1(1)} & \num{11} & \num{0.280(9)}\\
		\bottomrule
	\end{tabular}
\end{adjustbox}
	\caption{In Abhängigkeit der eingestellten Spannung aufgenommene Zeitdifferenzen für die jeweilige Anzahl an Perioden.
               Durchgeführt wurde die Messung an den Resonanzstellen beider Isotope bei einer Frequenz von \SI{100}{\kilo\hertz}.
               Aus den aufgenommenen Messwerten wurden die Periodendauern der Relaxtion bestimmt. \label{tab:periodendauern}}
\end{table}
