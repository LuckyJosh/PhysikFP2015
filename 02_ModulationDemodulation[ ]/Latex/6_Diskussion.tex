% !TeX root = Protokoll.tex

Die Versuchsergebnisse zeigen im allgemeinen eine plausible Übereinstimmung
mit den jeweiligen theoretischen Vorhersagen. 

Die durchgeführte Amplitudenmodulation lieferte in den
beiden Fällen (mit Trägerabstrahlung und Trägerunterdrückung)
den erwarteten zeitlichen Verlauf und das entsprechende 
Frequenzspektrum mit geringen Abweichungen. Bei den Abweichungen im Spektrum
handelt es sich um zusätzlich auftretende Frequenzen. Diese Frequenzen sind 
dabei auf die bereits in den Ausgangsspannungen vorhandenen Oberschwingungen
der Grundfrequenz zurückzuführen. Die Modulationsgrade die aus dem Spektrum 
und dem zeitlichen Verlauf bestimmt wurden weichen um etwa \SI{9}{\percent}
von einander ab und liegen beide in der Größenordnung des Modulationsgrades
den man aus den gewählten Eingangseinstellungen erhält.

Die Überprüfung der Abhängigkeit von der am Ausgang eines Ringmodulators gemessenen 
Gleichspannung von der Phase der Wechselspannung an den dessen Eingängen konnte mit 
nur geringen Abweichungen gezeigt und so die theoretische Vorhersage bestätigt werden. 

Die Demodulation der amplitudenmodulierten Spannung ergab für beide der verwendeten 
Methoden (Ringmodulator und Diode) eine zur Modulationsspannung proportionale Spannung,
die aufgrund der unterschiedlichen Laufzeiten in den Leitungen und den Widerständen 
in der verwendeten Schaltung einen Unterschied in Phase und Amplitude aufweist.

Die durchgeführte Frequenzmodulation zeigte ebenfalls nur geringe 
Abweichungen von dem erwarteten zeitlichen Verlauf. Eine ungenaue 
Einstellung der Phase zwischen den verwendeten Eingangsspannungen 
führte zusätzlich zu einer schwachen jedoch erkennbaren Amplitudenmodulation.

Die mit einem Schwingkreis aus der frequenzmodulierten Spannung
erzeugte amplitudenmodulierte Spannung zeigt Abweichungen von dem
erwarteten Verlauf einer Schwebung. Dies ist auf zusätzliche Frequenzen
in der Modulation zu begründen. Aufgrund der starken Abschwächung des Signals
war es außerdem nicht möglich die Grenzfrequenz des verwendeten Tiefpasses
zu verringern, sodass die Ergebnisspannung der Demodulation die doppelte 
Frequenz der Modulationsspannung aufweist.  


