% !TeX root = Protokoll.tex
\subsection{Messungen zum Linearverstärker}
In dieser Messreihe soll eine Schaltung nach \cref{fig:gegen_gekoppelter_invertierter_Linarverstärker} untersucht werden.
Dazu wird zunächst der Frequenzgang $f$ für verschiedene Verstärkungen $V'$ gemessen.
Weiter wird für eine Verstärkung $V'$ die Phase zwischen Eingangsspannung $U_\text{E}$ und Ausgangsspannung~$U_\text{A}$ für verschiedene Frequenzen $f$ untersucht.
\subsection{Vermessung des Amperemeters}
Es wird der Eingangswiederstand $r_\text{e}$ und die Leerlaufverstärkung $V$ eines Amperemeters bestimmt, die Schaltung wird nach \cref{fig:Amperemeter} aufgebaut.
Dazu werden die Eingangsspannung $U_\text{N}$, die Ausgangsspannung $U_\text{A}$ und die Generatorspannung $U_g$ in Abhängigkeit der Frequenz $f$ gemessen.
Dabei wird ein Widerstand $R_\text{N}$ mit \SI{10}{\kilo\ohm} verwendet.
\subsection{Umkehrintegrator und Umkehrdifferenzierer}
Es wird ein Umkehrintegrator nach \cref{fig:Umkehr_Integrator} aufgebaut.
Dabei wird die Frequenzabhängigkeit der Ausgangsspannung~$U_\text{A}$ untersucht, wenn eine Sinusspannung angelegt wird.
Weiter werden Bilder mithilfe eines Oszilloskops aufgenommen, wenn eine Rechtecks- und eine Dreiecksspannung anliegt.
Dies wird auch für einen Umkehrdifferenzierer untersucht, der wie in \cref{fig:Umkehr_Diff} aufgebaut wird.
\subsection{Schmitt-Trigger}
Als nächstes wird eine Schaltung nach \cref{fig:Schmitt-Trigger} untersucht.
Dazu wird ein Generator mit einer Sinusspannung an den Eingang angeschlossen und am Ausgangs wird ein Oszilloskop angeschlossen.
Es wird die Spannung des Generators von \SI{0}{\volt} hochgestellt, bis der Trigger anfängt zu kippen.
Damit wird die doppelte Betriebsspannung $2U_\text{B}$ ermittelt.
\subsection{Dreieckgenerator}
Es wird ein Dreieckgenerator nach \cref{fig:Dreieck} untersucht.
Mithilfe eines Oszilloskops kann die Zeitabhängigkeit überprüft werden.
Es werden die Frequenz und die Amplitude der Dreieckspannung gemessen.