\begin{table}[!h]
	\centering
\begin{adjustbox}{width=1\textwidth}
	\begin{tabular}{ccccc}
		\toprule
		Steigung & $y$-Achsenabschnitt & Grenzfrequenz & Verstärkung-Bandbreite & Leerlaufverstärkung\\
		$a$ & $b$ & $f_{\mathrm{g}}$/\si{\kilo\hertz} & 
		$f_{\mathrm{g}}V^{\prime}_{\mathrm{max}}$/\si{\kilo\hertz} & $V$\\
\midrule
		\num{-0.61(4)} & \num{0.52(6)} & \num{11(3)} & \num{12(3)} & \num{-15.000}\\
		\num{-0.19(7)} & \num{-0.43(8)} & \num{11(14)} & \num{4(5)} & \num{-0.142}\\
		\num{-0.32(6)} & \num{0.00(8)} & \num{24(18)} & \num{12(9)} & \num{-0.796}\\
		\num{-0.68(2)} & \num{0.83(2)} & \num{5.0(4)} & \num{16(1)} & \num{-720.000}\\
		\bottomrule
	\end{tabular}
\end{adjustbox}
	\caption{ Parameter der Ausgleichskurven der abfallenden Verstärkung, sowie die Grenzfrequenz und das 
Verstärkung-Bandbreiten-Produkt für jeder der vier Schaltungen. Bezeichnet werden die Parameter mit Steigung und 
y-Achsenabschnitt, da die Parameter in doppellogarithmischer Darstellung diese Bedeutung haben. \label{tab:gegengekoppleter_verstaerker_parameter}}
\end{table}
