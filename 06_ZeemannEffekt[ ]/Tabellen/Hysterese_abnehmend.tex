\begin{table}[!h]
	\centering
	\begin{tabular}{cccc}
		\toprule
		Stromstärke & magn. Flussdichte & Stromstärke & magn. Flussdichte\\
		$I$/\si{\ampere} & $B$/\si{\milli\tesla} & $I$/\si{\ampere} & $B$/\si{\milli\tesla}\\
\midrule
		\num{0.0(5)} & \num{8(1)} & \num{9.0(5)} & \num{538(1)}\\
		\num{1.0(5)} & \num{75(1)} & \num{10.0(5)} & \num{591(1)}\\
		\num{2.0(5)} & \num{144(1)} & \num{11.0(5)} & \num{646(1)}\\
		\num{3.0(5)} & \num{202(1)} & \num{12.0(5)} & \num{701(1)}\\
		\num{4.0(5)} & \num{261(1)} & \num{13.0(5)} & \num{775(1)}\\
		\num{5.0(5)} & \num{315(1)} & \num{14.0(5)} & \num{808(1)}\\
		\num{6.0(5)} & \num{371(1)} & \num{15.0(5)} & \num{856(1)}\\
		\num{7.0(5)} & \num{435(1)} & \num{16.0(5)} & \num{887(1)}\\
		\num{8.0(5)} & \num{490(1)} & \num{17.0(5)} & \num{925(1)}\\
		\bottomrule
	\end{tabular}
	\caption{Messwerte der magnetischen Flussdichte, die in Abhängigkeit der angelegten Stromstärke
                        aufgenommen wurden, um die Hysterese des verwendeten Magneten zu bestimmen.
                        In dieser Messung wurde die Stromstärke sukzessive verringert. \label{tab:hysterese_abnehmend}}
\end{table}
