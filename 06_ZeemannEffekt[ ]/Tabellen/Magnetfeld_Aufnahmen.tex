\begin{table}[!h]
	\centering
	\begin{tabular}{lcc}
		\toprule
		Messung & Stromstärke & magn. Flussdichte\\
		& $I$/\si{\ampere} & $B$/\si{\milli\tesla}\\
\midrule
		rot, blau&\num{0.0(5)} & \num{29(30)}\\
		blau $\sigma$ &\num{5.0(5)} & \num{309(29)}\\
		rot $\sigma$ &\num{10.0(5)} & \num{590(29)}\\
		blau $\pi$ &\num{17.0(5)} & \num{983(30)}\\
		\bottomrule
	\end{tabular}
	\caption{Stromstärken und die korrespondierenden magnetischen Flussdichten,
                    bei denen die Spektrallinien aufgenommen wurden.
                    Die magnetischen Flussdichten wurden dabei mit der zuvor bestimmten Fit-Gerade der
                    Hysterese-Messung berechnet. \label{tab:strom_magnetfeld}}
\end{table}
