\begin{table}[!h]
	\centering
	\subfloat{
	\begin{tabular}{cccc}
		\toprule
		\multicolumn{2}{c}{1.Messreihe} & \multicolumn{2}{c}{2.Messreihe} \\\cmidrule(rl){1-2}\cmidrule(rl){3-4}
		Druck & gemittelte Zeit & Druck & gemittelte Zeit\\
		$p$/\si{mbar} & $\bar{t}$/\si{s} & $p$/\si{mbar} & $\bar{t}$/\si{s}\\
\midrule
		\num{0.20(6)} & \num{12(1)} & \num{0.4(1)} & \num{17(1)}\\
		\num{0.4(1)} & \num{64(1)} & \num{0.6(2)} & \num{46(1)}\\
		\num{0.6(2)} & \num{128(1)} & \num{0.8(2)} & \num{72(1)}\\
		\bottomrule
%	\end{tabular}
%	}\\
%	\subfloat{
%	\begin{tabular}{cccc}
		\toprule
		\multicolumn{2}{c}{3.Messreihe} & \multicolumn{2}{c}{4.Messreihe} \\\cmidrule(rl){1-2}\cmidrule(rl){3-4}
		Druck & gemittelte Zeit & Druck & gemittelte Zeit\\
		$p$/\si{mbar} & $\bar{t}$/\si{s} & $p$/\si{mbar} & $\bar{t}$/\si{s}\\
		\midrule
		\num{0.6(2)} & \num{9(1)} & \num{2.0(6)} & \num{12(1)}\\
		\num{0.8(2)} & \num{18(1)} & \num{4(1)} & \num{34(1)}\\
		\num{1.0(3)} & \num{42(1)} & \num{6(2)} & \num{55(1)}\\
		\bottomrule
	\end{tabular}
	}
	
	\caption{Werte der Leckratenmessung unter Verwendung der Drehschieberpumpe.
                        Neben den gemessenen Drücken sind die gemittelten Zeiten angegeben. 
                        Dabei sind die Fehler der Zeiten systematischen Ursprungs und wurden 
                        daher durch das Mitteln nicht reduziert. \label{tab:Leckratenmessung_Drehschieber_0}}
\end{table}
