% !TeX root = Protokoll.tex
\subsection{Definition des Vakuums}
Ein Vakuum wird allgemein definiert, als Abwesenheit von Materie.
Dies bezieht sich auf Gase oder Flüssigkeiten. Um die Größe eines Vakuums zu messen, wird der Druck verwendet. Umso geringer der Druck umso besser das Vakuum.\\
Ein Vakuum beginnt bei \SI{300}{\milli\bar}, dass wird Grobvakuum genannt.\\
In einem Druckbereich von \SI{1}{\milli\bar} bis zu \SI{e-3}{\milli\bar} liegt ein Feinvakuum vor.\\
Das Hochvakuum beginnt bei \SI{e-3}{\milli\bar} und reicht bis zu bis zu \SI{e-7}{\milli\bar}. Alle Drücke die da drunter liegen, beschreiben ein Ultrahochvakuum.\\
Eine weitere Größe, die behilflich bei dem Verständnis eines Vakuums ist, ist die mittlere freie Weglänge
\begin{align}
\Lambda = \frac{1}{\sqrt{2}\pi D^2n},
\end{align}
dabei ist $n$ die Teilchenzahldichte und $D$ der Durchmesser der Gasmoleküle. Mithilfe der Teilchenzahldichte 
\begin{align}
n=\frac{N}{V}
\end{align}
und der idealen Gasgleichung
\begin{align}
p\cdot V = N k_B T \label{eq:IdealGas}
\end{align}
für einen Druck $p$, ein Volumen $V$, einer Teilchenzahl $N$ und einer Temperatur $T$, wobei $k_B$ die Boltzmann-Konstante ist.
Kann die mittlere freie Weglänge als
\begin{align}
\Lambda = \frac{k_BT}{\sqrt{2}\pi D^2p}
\end{align} 
Diese Größe Beschreibt die durchschnittliche Strecke die ein Molekül zurücklegt, bis es mit einem anderen Zusammen stößt. Bei geringer werdendem Druck, wächst die mittlere freie Weglänge.

\subsection{Herleitung der $p(t)$-Kurve}
Unter Annahme der idealen Gasgleichung (\ref{eq:IdealGas})
kann eine Näherung für den Zeitlichen Druckverlauf gemacht werden, für einer konstante Temperatur.
Durch ableiten nach der Zeit wird die Gleichung zu
\begin{align}
\frac{d\left(p\cdot V \right)}{dt}=p\frac{dV}{dt}+V\frac{dp}{dt}=0
\end{align}
Durch umstellen, Integration und Einführung des Saugvermögens $S=\frac{dV}{dt}$, wird die Gleichung zu
\begin{align}
p(t) = p_0 \exp\left(-\frac{S}{V_0}t\right).
\end{align}
Da der Druck allerdings gegen einen Enddruck $p_E$ läuft, wird folgende Lösung verwendet.
\begin{align}
p(t) = \left( p_0-p_E \right) \exp\left(-\frac{S}{V_0}t\right) + p_E
\end{align}

\subsection{Strömungen}