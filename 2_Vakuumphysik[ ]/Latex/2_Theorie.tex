% !TeX root = Protokoll.tex
\subsection{Definition des Vakuums}
Ein Vakuum wird allgemein definiert, als Abwesenheit von Materie.
Dies bezieht sich auf Gase oder Flüssigkeiten. Um die Größe eines Vakuums zu messen, wird der Druck verwendet. Umso geringer der Druck umso besser das Vakuum.\\
Ein Vakuum beginnt bei \SI{300}{\milli\bar}, dass wird Grobvakuum genannt.\\
In einem Druckbereich von \SI{1}{\milli\bar} bis zu \SI{e-3}{\milli\bar} liegt ein Feinvakuum vor.\\
Das Hochvakuum beginnt bei \SI{e-3}{\milli\bar} und reicht bis zu bis zu \SI{e-7}{\milli\bar}. Alle Drücke die da drunter liegen, beschreiben ein Ultrahochvakuum.\\
Eine weitere Größe, die behilflich bei dem Verständnis eines Vakuums ist, ist die mittlere freie Weglänge
\begin{align}
\Lambda = \frac{1}{\sqrt{2}\pi D^2n},
\end{align}
dabei ist $n$ die Teilchenzahldichte und $D$ der Durchmesser der Gasmoleküle. Mithilfe der Teilchenzahldichte 
\begin{align}
n=\frac{N}{V}
\end{align}
und der idealen Gasgleichung
\begin{align}
p\cdot V = N k_B T \label{eq:IdealGas}
\end{align}
für einen Druck $p$, ein Volumen $V$, einer Teilchenzahl $N$ und einer Temperatur $T$, wobei $k_B$ die Boltzmann-Konstante ist.
Kann die mittlere freie Weglänge als
\begin{align}
\Lambda = \frac{k_BT}{\sqrt{2}\pi D^2p}
\end{align} 
Diese Größe Beschreibt die durchschnittliche Strecke die ein Molekül zurücklegt, bis es mit einem anderen Zusammen stößt. Bei geringer werdendem Druck, wächst die mittlere freie Weglänge.

\subsection{Herleitung der $p(t)$-Kurve}
Unter Annahme der idealen Gasgleichung (\ref{eq:IdealGas})
kann eine Näherung für den Zeitlichen Druckverlauf gemacht werden, für einer konstante Temperatur.
Durch ableiten nach der Zeit wird die Gleichung zu
\begin{align}
\frac{d\left(p\cdot V \right)}{dt}=p\frac{dV}{dt}+V\frac{dp}{dt}=0
\end{align}
Durch umstellen, Integration und Einführung des Saugvermögens $S=\frac{dV}{dt}$, wird die Gleichung zu
\begin{align}
p(t) = p_0 \exp\left(-\frac{S}{V_0}t\right).
\end{align}
Da der Druck allerdings gegen einen Enddruck $p_E$ läuft, wird folgende Lösung verwendet.
\begin{align}
p(t) = \left( p_0-p_E \right) \exp\left(-\frac{S}{V_0}t\right) + p_E
\end{align}

\subsection{Pumpvorgang}
Ein Gefäß, in dem ein Vakuum erzeugt werden soll, wird Rezipient genannt. Das Gas was abgepumpt wird strömt in Form von Viskose Strömungen aus dem Rezipient, wenn der Druck dem Grobvakuum zugehörig ist. Bei einem Hochvakuum oder Ultrahochvakuum geschieht dies über molekulare Strömungen.
\subsubsection{Viskose Strömungen}
Bei Viskose Strömungen überwiegen die Stöße der Moleküle gegenüber den Stößen mit der Außenwand des Rezipienten.
\begin{align}
 \Lambda \ll 2 r
\end{align}
Das heißt die mittlere freie Weglänge ist viel Größer als der Durchmesser des Rezipienten oder der Rohre. Weiter bedeutet es, dass das die Strömung als kontinuierlich betrachtet werden kann, sie wird mithilfe von laminare oder turbulenten Strömungen beschrieben.
\subsubsection{Molekulare Strömungen}
Bei molekularen Strömungen überwiegen die Stöße mit der Wand und den Molekülen, weil sich kaum noch Moleküle im Rezipienten befinden. Bei einem Stoß mit der Wand werden die Moleküle nicht reflektiert sondern haften an der Wand und lösen sich nach geringer Zeit wieder. Das Haften an der Wand wird als Adsorption bezeichnet und das Verlassen Desorption.
Bei der Desorption werden die Moleküle statistisch verteilt ausgelöst. Die Strömung entsteht dadurch das die Teilchen im Durchschnitt sich in eine Richtung bewegen.
\subsubsection{Leitwert}
Der Leitwert $L$ der als der Kehrwert des Strömungswiderstand $W$ Definiert ist wird berechnet durch
\begin{align}
L=\frac{1}{W} = \frac{q_{pV}}{\Delta p}
\end{align}
Dabei ist $q_{pV}$ der Strom des Volumens und $\Delta p$ die Druckdifferenz an den Bauteilenden. Für in reihe verbaute Bauteile  gilt, dass der inverse gesamte Leitwert $L_\text{ges}$ sich aus der summe der Kehrwerte der Leitwerte berechnet.
 \begin{align}
 \frac{1}{L_\text{ges}}=\frac{1}{L_1}+\frac{1}{L_2}+...+\frac{1}{L_N}
 \end{align}
 
\subsection{Pumpen}
Ein Vakuum in einem Rezipienten wird mithilfe von Pumpen verwirklicht. Je nach gewünschtem Vakuum werden unterschiedliche Pumpen benötigt. Für Grob- oder Feinvakuen wird zum Beispiel eine Drehschiebepumpe verwendet und für Hochvakuen werden Turbomolekularpumpen verwendet.
\subsubsection{Drehschieberpumpe}
Die Drehschiebepumpe besteht aus einem Hohlzylinder mit einem Zulauf und einem Ablauf. Ein Roter befindet sich exzentrisch im inneren des Zylinders, an dem mehrere Schieber befestigt sind, die mithilfe von Federn an die außen Wand gedrückt werden und dadurch das innere in mehrere Volumina aufteilen. Durch den Rotor verändern sich die Volumina. Bei dem Zulauf wird das Volumen größer wodurch es dem Rezipienten Gas entzieht. Das Volumen wird dann vom Zulauf getrennt und überschreitet das Maximum. Das Volumen verringert sich weiter und der Ablauf wird zugefügt und das Gas wird somit aus der Pumpe her raus geführt.

\subsubsection{Turbomolekularpumpe}
Die Turbomolekularpumpe oder auch Turbopumpe nutzt den Effekt der Adsorption aus. Wenn das Gas an einer Oberfläche haftet und sich diese bewegt, dann hat das Molekül bei der Desorption eine bevorzugte Austrittsrichtung. Dabei besitzt eine Turbomolekularpumpe eine Reihe von Rotoren und Statorschafeln um diesen Effekt zu verbessern. Die Geschwindigkeit der Rotoren muss dazu so groß sein wie die Geschwindigkeit der Gasmoleküle, sowie muss eine Molekulare Strömung vorliegen damit die Moleküle den Impuls behalten.
\subsection{Messgeräte}
Der Druck wird mit verschiedenen Messgeräten gemessen, weil die unterschiedlichen Vakuen anders Vermessen werden müssen.
\subsubsection{Pirani-Vakuummeter}
Mithilfe der Wärmeleitfähigkeit von Gasen, kann der Druck bestimmt werden. Ein Draht innerhalb eines Hohlzylinders, wird auf eine Konstante Temperatur erhitzt. Das Gas ermöglicht einen Wärmestrom vom Draht zur Hülle. Der Wärmestrom ist Proportional zum Druck und wenn die Temperatur des Drahtes konstant ist die Heizleistung Druckabhängig. Die Heizleistung und der Druck haben einen exponentiellen Zusammenhang, was in eine Logarithmische Skala resultiert. Der Funktionsbereich geht vom Grobvakuum bis ins Hochvakuum.
\subsubsection{Kaltkathoden-Vakuummeter}



