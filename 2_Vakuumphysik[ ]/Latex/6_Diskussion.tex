% !TeX root = Protokoll.tex
Im Folgenden werden die in der Auswertung gewonnenen Ergebnisse noch einmal 
zusammen gefasst und abschließend diskutiert, um eine Aussage über deren Plausibilität 
treffen zu können.\\

Für die Messung der Evakuierungskurve der Drehschieberpumpe ergibt sich zunächst ein 
Verlauf der einer exponentiell abfallenden Kurve ähnlich sieht. Durch die Logarithmierung
des Druckes erhält man jedoch nicht nur eine lineare Funktion sondern mehrere Druckereiche
in denen ein linearer Verlauf zu erkennen ist. Dies entspricht der theoretischen Beschreibung,
da das Saugvermögen der Pumpen über den gesamten Druckbereich nicht konstant ist.\\
Die Leckratenmessung der Drehschieberpumpe ergibt, wie von der Theorie vorhergesagt, für jeden
Gleichgewichtsdruck und damit jede Messreihe eine lineare Funktion.\\
Aus beiden Messungen kann durch lineare Ausgleichsrechnung die Steigungen der jeweiligen 
Geraden ermittelt und damit das Saugvermögen bei dem herrschenden Druck bestimmt werden.
Die gegen den Druck aufgetragenen Werte für das Saugvermögen zeigen zunächst den in der 
Literatur \cite{Pfeifer} angegebenen Verlauf. Dieser beschreibt einen erst flachen und dann
steilen Anstieg und eine Annäherung an einen Sättigungswert. Die in diesem Versuch berechneten 
Werte steigen auch zunächst bis zu einem Maximum an und fallen nach diesem wieder ab.
Das Abfallen im niedrigen Druckbereich ist durch den Versuchsaufbau zu erklären, da bei 
hohen Drücken der Desorptionsstrom der Gasteilchen von den Innenflächen des Rezipienten überwiegt
und die Pumpe in diesem Druckberich gegen diesen arbeiten muss. Vergleichbares gilt für den 
Bereich sehr geringer Drücke, in diesen ist die Innenfläche des Rezipienten so sauber, 
dass diese selbst als Pumpe wirkt und somit das Saugvermögen der Drehschieberpumpe reduziert.\\
Die Herstellerangabe \cite{Plakette} zum Saugvermögen der Drehschieberpumpe liegt bei 
\begin{empheq}{align}
	S_{\mathrm{soll,dreh}} &=\SI{4}{\m\cubed\per\hour}\\
	 &\approx \SI{1.11}{\l\per\s}.\notag
\end{empheq}
Das Maximum der in diesem Versuch bestimmten Werte ergab sich zu\\
$S_{\mathrm{max,dreh}} = \SI{1.2(1)}{\l\per\s}$, sodass der theoretische Wert
im Rahmen des Fehler liegt.\\

In den Messungen mit der Turbopumpe lassen sich Grundsätzliche die gleichen 
Beobachtungen machen wie bei der Durchführung mit der Drehschieberpumpe.
Auch hier ergeben sich bei der logarithmierten Evakuierungskurve Druckbereiche mit linearem 
Verlauf. Und auch die Leckratenmessung führt zu Messwerten, die für jeden eingestellten 
Gleichgewichtsdruck einen linearen Verlauf zeigen.\\
Der Unterschied zu der Messung mit der Drehschieberpumpe wird jedoch in der Darstellung des 
Saugvermögens in Abhängigkeit des Druckes sichtbar. In dieser zeigt sich, dass 
die Werte die aus der Evakuierungskurve gewonnen wurden sehr stark von denen aus der
Leckratenmessung abweichen. Eine mögliche Erklärung dieser Diskrepanz ist, die bereits erwähnte
Pumpwirkung der Innenfläche des Rezipienten. Da die Turbopumpe in einem Druckbereich betrieben 
wurden, der zwei bis drei Größenordnungen geringer ist als bei der Drehschieberpumpe ist 
dieser Effekt noch viel stärker zu erwarten.\\
Im Gegensatz dazu ist bei der Messung der Leckrate, bei der Luft eingelassen wird, diese Pumpwirkung
des Rezipienten der Turbopumpe nicht entgegengesetzt, sodass in diesem Versuch die Summe der 
beiden Saugvermögen (von Turbopumpe und Rezipient) bestimmt wird.\\
Beim Vergleich der Messwerte mit der Herstellerangabe \cite{Hefter} von
\begin{empheq}{equation}
	S_{\mathrm{soll,turbo}} = \SI{77}{\l\per\s},
\end{empheq}
zeigt sich ein weiterer großer Unterschied. Dieser ist wiederum durch den Aufbau zu erklären, 
da direkt am Ansaugstutzen der Turbopumpe ein Flansch angebracht war, der aufgrund seines 
geringeren Durchmessers und damit geringeren Leitwertes $L_{\mathrm{flansch}}$ das Saugvermögen der Turbopumpe,
wie durch \eqref{eq:Leitwert_Reihenschaltung} beschrieben, stark verringert.
Geht man von einer Verringerung um den Faktor $10$ aus, so erhält man das effektive Saugvermögen
\begin{empheq}{equation}
S_{\mathrm{eff,turbo}} = \SI{7.7}{\l\per\s}.
\end{empheq}
Betrachtet man nun die Messwerte aus der Leckratenmessung zeigt sich, dass der Effektivwert im Fehlerbereich
von drei der vier Messwerte liegt.


