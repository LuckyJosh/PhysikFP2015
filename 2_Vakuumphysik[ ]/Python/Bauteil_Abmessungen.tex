\begin{table}[!h]
	\centering
	\begin{tabular}{ccccccc}
		\toprule
		Bauteil Nummer & Durchmesser (groß) & Durchmesser (klein) & Länge (groß) & Länge (klein) & Volumen (offen) & Volumen (geschlossen)\\
		$$ & $d_\mathrm{g}$/\si{mm} & $d_\mathrm{k}$/\si{mm} & $l_\mathrm{g}$/\si{mm} & $l_\mathrm{k}$/\si{mm} & $V$/\si{cm\cubed} & $V$/\si{cm\cubed}\\
\midrule
		\num{1} & \num{38.5(1)} & \num{-1} & \num{130(5)} & \num{-1} & \num{151(6)} & \num{-1}\\
		\num{2} & \num{151(5)} & \num{-1} & \num{480(10)} & \num{-1} & \num{8595(639)} & \num{-1}\\
		\num{3} & \num{26(1)} & \num{-1} & \num{75(5)} & \num{-1} & \num{39(4)} & \num{-1}\\
		\num{4} & \num{34(2)} & \num{-1} & \num{75(5)} & \num{-1} & \num{69(9)} & \num{-1}\\
		\num{5} & \num{38(5)} & \num{-1} & \num{400(5)} & \num{-1} & \num{453(120)} & \num{-1}\\
		\num{6} & \num{15(2)} & \num{-1} & \num{240(5)} & \num{-1} & \num{42(11)} & \num{-1}\\
		\num{7} & \num{15(2)} & \num{-1} & \num{240(5)} & \num{-1} & \num{42(11)} & \num{-1}\\
		\num{8} & \num{15(2)} & \num{-1} & \num{440(5)} & \num{-1} & \num{77(21)} & \num{-1}\\
		\num{9} & \num{40(1)} & \num{-1} & \num{52(5)} & \num{-1} & \num{65(7)} & \num{-1}\\
		\num{10} & \num{40(1)} & \num{-1} & \num{129(5)} & \num{46(1)} & \num{219(13)} & \num{-1}\\
		\num{11} & \num{11.0(1)} & \num{-1} & \num{80(5)} & \num{36(5)} & \num{11.0(7)} & \num{-1}\\
		\num{12} & \num{40(1)} & \num{-1} & \num{110(20)} & \num{60(10)} & \num{213(30)} & \num{-1}\\
		\num{13} & \num{39.5(1)} & \num{16.0(1)} & \num{130(5)} & \num{40(5)} & \num{175(7)} & \num{-1}\\
		\num{14} & \num{11.0(1)} & \num{11.0(1)} & \num{80(5)} & \num{70(5)} & \num{20(1)} & \num{6(1)}\\
		\num{15} & \num{15(1)} & \num{-1} & \num{77(5)} & \num{-1} & \num{13(2)} & \num{20(3)}\\
		\bottomrule
	\end{tabular}
	\caption{Geometrische Abmessungen aller verwendeten Bauteile und die aus diesen berechneten Volumina.
                  Die Nummerierung entspricht der auf ABBILDUNG???. \label{tab:Bauteile_Maße}}
\end{table}
