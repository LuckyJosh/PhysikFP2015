% !TeX root = Protokoll.tex

Die im ersten Versuchsteil beobachteten Signalbilder wie in \cref{fig:signalbild_rf100khz} 
entsprechen der erwarteten Gestalt. Die aus dem Verhältnis der Größen der Minima bestimmten 
Anteile der beiden Isotope weichen mit
\begin{empheq}{equation*}
P_{\mathrm{Rb87}} = P_{{}^{87}\!\mathrm{Rb}} =  \SI{36.4(7)}{\percent}
 \qquad P_{\mathrm{Rb85}} =  \SI{63.6(7)}{\percent}

\end{empheq}
leicht von der natürlich vorkommenden Verteilung 
\begin{empheq}{equation*}
P_{\mathrm{Rb87},\mathrm{lit}} = \SI{27.8}{\percent} \qquad P_{\mathrm{Rb85},\mathrm{lit}} = \SI{72.2}{\percent}
\end{empheq}
ab. Zu vermuten ist, dass ein größerer Anteil von ${}^{87}\!$Rb vorliegt, um das entsprechenden 
Minimums im Signal zu vergrößern.
Die bestimmten Lande-Faktoren  
\begin{empheq}{equation*}
g_{F,\mathrm{Rb87}} = g_{{}^{87}\!\mathrm{Rb}} =  \num{0.498(2)}
\qquad g_{F,\mathrm{Rb85}} = g_{{}^{85}\!\mathrm{Rb}} =  \num{0.330(1)}

\end{empheq}  
entsprechen im Rahmen der Messfehler und mit nur geringen Abweichungen den Literaturwerten
\begin{empheq}{equation*}
g_{F,\mathrm{Rb87},\mathrm{lit}} = \frac{1}{2} \qquad g_{F,\mathrm{Rb85},\mathrm{lit}} = \frac{1}{3}.
\end{empheq}

In gleicher Weise sind auch die Abweichungen der bestimmten Kernspins 
\begin{empheq}{equation*}
I_{\mathrm{Rb87}} = I_{{}^{87}\!\mathrm{Rb}} =  \num{1.509(8)}
\qquad
I_{\mathrm{Rb85}} = I_{{}^{85}\!\mathrm{Rb}} =  \num{2.53(1)}

\end{empheq}
von den Literaturwerten nur sehr gering
\begin{empheq}{equation*}
I_{\mathrm{Rb87},\mathrm{lit}} = \frac{3}{2}\qquad
I_{\mathrm{Rb85},\mathrm{lit}} = \frac{5}{2}
\end{empheq}
und es war eine eindeutige Identifikation der Minima im Signalbild mit den Isotopen möglich.

Die in diesem Versuch bestimmte horizontale Komponente des Erdmagnetfelds wurde zu
\begin{empheq}{equation}
	B_{\mathrm{Erde},\mathrm{hor}} =  \num{1.90(3)e-05}

\end{empheq}
bestimmt. Da sich die Horizontalkomponente $B_{\mathrm{Erde},\mathrm{hor}}$ aus dem Gesamtmagnetfeld $B_{\mathrm{Erde}} = \SI{47}{\micro\tesla}$\cite{GGU} durch die Gleichung $B_{\mathrm{Erde},\mathrm{hor}} = B_{\mathrm{Erde}} \cdot \cos(\varphi)$,
mit dem Inklinationswinkel $\varphi$, ergibt erhält man für diesen den Wert   
\begin{empheq}{equation}
\varphi = \SI{66.1(4)}{\degree}.
\end{empheq}\\
Dieser Wert entspricht mit guter Genauigkeit dem in Dortmund auf dem Breitengrad \SI{51.5}{\degree}N vorliegenden,
wie in der Grafik in \cite{GFZ} zu erkennen ist. Damit stellt dieser Versuch auch eine sehr genaue Möglichkeit zur 
Verfügung, um das Erdmagnetfeld zu bestimmen.\\ 
Die Beobachtungen im zweiten Versuchsteil decken sich ebenfalls mit den Erwartungen. 
Die Anpassung der Ausgleichskurven an den ansteigenden Teil des Signalbildes bestätigt 
das theoretisch vorhergesagte exponentielle Anwachsen. 
Der Quotient der Parameter $b_{i}$ der hyperbolischen Ausgleichskurven, die an die Messwerte der Periodendauer in 
Abhängigkeit der RF-Spannung angepasst wurden liefern für die erste Anpassung bei der alle Messwerte 
verwendet wurden den Wert $ \num{23(85)}
$ und für die Anpassung 
in der der erste Messwert ausgelassen wurde den Wert \num{1.4(6)}
. 
An diesen Ergebnissen wird noch einmal deutlich, dass der erste Messwert eine Fehlmessung darstellt und
nur durch aus lassen dieses Messwertes ein Ergebnis erzielt werden kann, welches im Rahmen des 
Fehlers dem theoretischen Wert \num{1.5} entspricht.\\

Die Größe des quadratischen Zeeman-Effekts lässt sich mit Hilfe von \cref{eq:zeeman_effekt_quadratisch}
und den in diesem Versuch aufgenommenen Messwerten schätzen. Da der quadratische Zeeman-Effekt für große
Magnetfelder $B$ relevant wird, wird für die Schätzung der Wert $B = \SI{0.25}{\milli\tesla}$ gewählt,
welcher größer ist als alle in diesem Versuch erzeugten Magnetfelder, sich jedoch noch in der selben 
Größenordnung befindet. Für die Lande-Faktoren der beiden Rubidium-Isotope  werden die Literaturwerte 
verwendet. Für die Quantenzahl $m_{F}$ wird wiederum jeweils das die größtmögliche gewählt,
$m_{F,\mathrm{Rb85}} = 3$ und $m_{F,\mathrm{Rb87}} = 2$. Die Hyperfein-Aufspaltung ist 
mit $\Delta E_{\mathrm{Hy},\mathrm{Rb87}} = \SI{4.53e-24}{\joule}$ und $\Delta E_{\mathrm{Hy},
\mathrm{Rb85}} = \SI{2.01e-24}{\joule}$ angegeben\,\cite{V21}. Aus diesen Werten ergibt sich der quadratische
Zeeman-Effekt zu \\$U_{\mathrm{HF},\mathrm{Rb87}}$ =  \SI{-8.9e-31}{\joule}

und $U_{\mathrm{HF},\mathrm{Rb85}}$ =  \SI{-1.5e-30}{\joule}
.
Für beide Isotope liegt der quadratische Zeeman-Effekt damit mehr als zwei Größenordnungen unter dem 
linearen Term und ist damit in diesem Versuch vernachlässigbar. 
 