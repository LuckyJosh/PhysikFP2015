\subsection{Fehlerrechnung}
Im folgenden Abschnitt werden die, für die Auswertung der aufgenommenen Daten
verwendeten Gleichungen aufgezeigt und erläutert.\\
Der Mittelwert aus mehreren Ergebnissen einer Messung 
wurde mit Hilfe von 
\begin{empheq}{equation}
	\mean{x} = \frac{1}{n}\sum_{i = 0}^{n}x_i
	\label{eq:Mittelwert}
\end{empheq}
berechnet.
Für die Berechnung der statistischen Abweichung wurde die folgende Gleichung verwendet:
\begin{empheq}{equation}
\sigma_{x} = \sqrt{\frac{1}{n}\sum_{i = 0}^{n}(x_i-\mean{x})^2}.
\label{eq:Mittelwert_Std}
\end{empheq}

Für die Fehlerfortpflanzung der Messunsicherheiten wurde die 
gaußsche Fehlerfortpflanzung verwendet.
Damit berechnet sich der Fehler $\sigma_y$ einer Größe $y = y(\va{x})$, mit den Messgrößen $\dim{\va{x}} = n$, wie folgt:
\begin{empheq}{equation}
\sigma_{y} = \sqrt{\sum_{i = 0}^{n}{\qty(\pdv{f}{x_i}\sigma_{x_i})^2}}.
\label{eq:Fehlerforpflanzung}
\end{empheq}

Die in dieser Auswertung verwendeten Fehlergleichungen befinden sich im folgenden \cref{sec:Fehlergleichungen}.

Die relative Abweichung eines Messergebnisses $x$ vom gegebenen Theoriewert 
$x_{\mathrm{theo}}$ wurde mit folgender Gleichung berechnet:
\begin{empheq}{equation}
\Delta_{\mathrm{rel}}x = \frac{\envert{x - x_{\mathrm{theo}}}}{x_{\mathrm{theo}}}.
\label{eq:Fehler_relativ}
\end{empheq}

Die in der Auswertung angefertigten Regressionskurven wurden mit Hilfe der \emph{Python}-Bibliothek \emph{scipy} \cite{SciPy}
durchgeführt.

\newpage
\subsection{Fehlerfortpflanzung}\label{sec:Fehlergleichungen}

Für die Berechnung des Fehlers der logarithmierten Druckmesswerte der Form\\ $P = \ln(\frac{p - p_e}{p_0 - p_e})$
wurde die folgenden Gleichung verwendet.
\begin{empheq}{equation}
	\sigma_{P}=\sqrt{\frac{\sigma_{p}^{2}}{\left(p - p_{e}\right)^{2}} + \frac{\sigma_{p_{0}}^{2}}{\left(p_{0} - p_{e}\right)^{2}} + \frac{\sigma_{p_{e}}^{2} \left(p_{0} - p_{e}\right)^{2}}{\left(p - p_{e}\right)^{2}} \left(\frac{p - p_{e}}{\left(p_{0} - p_{e}\right)^{2}} - \frac{1}{p_{0} - p_{e}}\right)^{2}}
\end{empheq}

Der Fehler des Saugvermögens nach  Gleichung \eqref{eq:Saugvermoegen_Leckrate} ergibt sich mit der
Steigung $a = \frac{\Delta p}{\Delta t}$  aus der Fehlergleichung:
\begin{empheq}{equation}
\sigma_{S}=\sqrt{\frac{V^{2} a^{2}}{p_{g}^{4}} \sigma_{p_{g}}^{2} + \frac{V^{2} \sigma_{a}^{2}}{p_{g}^{2}} + \frac{a^{2} \sigma_{V}^{2}}{p_{g}^{2}}}
\end{empheq}

 