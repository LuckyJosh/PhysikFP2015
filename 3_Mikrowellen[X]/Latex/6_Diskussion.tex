% !TeX root = Protokoll.tex

Im folgenden Abschnitt werden die in der Auswertung erhaltenen 
Ergebnisse noch einmal abschließend diskutiert und auf ihre Plausibilität hin 
überprüft.\\

Die im ersten Versuch aufgenommenen Messwerte zu Reflektorspannung und 
Amplitude die in zeigen mit Einschränkungen den zu erwartenden Verlauf.
Der Theorie entsprechend erhält man für geringere Reflektorspannungen 
Moden mit höheren Frequenzen. Jedoch entspricht der Verlauf der Amplitude von 
kleinen zu großen Frequenzen nicht dem theoretischen Verlauf in \cite{V53}, da die Amplitude 
zunächst ansteigt und erst danach wieder abfällt. 

Die Bestimmung der Frequenz mittels der Wellenlänge im Hohlleiter $\lambda_g$
lieferte mit\\ $f = \SI{909(3)e7}{Hz} $ ein Ergebnis, welches mit einer 
relativen Abweichung von\\ $\Delta_{rel} f = \SI{0.6}{\percent}$ zum gemessenen 
Wert als sehr genau und plausibel einzustufen ist.

Der Vergleich der Dämpfungen die am SWR-Meter und von der Eichkurve des 
Dämpfungsglieds abgelesen wurden zeigt für beide Kurven einen ähnlichen 
Verlauf, welcher sich nur in der Größe der Funktionswerte unterscheidet.
Daraus lässt sich schließen, dass auch die Dämpfungen die durch das 
SWR-Meter bestimmt wurdem plausible Ergebnisse sind.

Die drei Messmethoden des Stehwellenverhältnises lieferten für die Sondentiefe von\\ $s = \SI{9.00(1)}{\mm}$ die drei Ergebnisse:
\begin{empheq}{align*}
	\mathrm{SWR}_{\mathrm{SWR-Meter}} &= \num{9.00} \\
	\mathrm{SWR}_{\mathrm{3dB}} &= \num{ 11.5(2)}\\
	\mathrm{SWR}_{\mathrm{Abschwächer}} &= \num{ 7.5(7)}
\end{empheq}
Im Vergleich mit der Ablesung vom SWR-Meter erhält man die relativen 
Abweichungen $\Delta_{rel} \mathrm{SWR}_{\mathrm{3dB}} = \SI{28}{\percent}$
und $\Delta_{rel} \mathrm{SWR}_{\mathrm{Abschwächer}} = \SI{17}{\percent}$.
Somit weichen die Messwerte mitunter stark von einander ab, liegen jedoch 
alle in einer plausiblen Größenordnung. Die Abweichung der
Abschwächermethode ist mit dem verwendeten Abschwächer begründbar, da
bei diesem Versuch kein Präzisionsabschwächer verwendet wurde.
Eine Begründung für die Abweichung der 3dB-Methode ist die 
Lokalisation der 3dB-Punkte, welche durch das abrupte Auftreten des Vollausschlags am SWR-Meter erschwert wurde.
  
