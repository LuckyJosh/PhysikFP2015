\begin{table}[!h]
	\centering
	\begin{tabular}{ccccc}
		\toprule
		Reflektorspannung & Reflektorspannung & Reflektorspannung & Amplitude & Frequenz\\
		$U_0$/\si{V} & $U_1$/\si{V} & $U_2$/\si{V} & $A$/\si{Teilstrich} & $f_0$/\si{MHz}\\
\midrule
		\num{155(5)} & \num{140(5)} & \num{170(5)} & \num{5.8(2)} & \num{9143(1)}\\
		\num{235(5)} & \num{220(5)} & \num{245(5)} & \num{5.3(2)} & \num{9142(1)}\\
		\num{100(5)} & \num{90(5)} & \num{110(5)} & \num{4.6(2)} & \num{9148(1)}\\
		\bottomrule
	\end{tabular}
	\caption{Messwerte für die Reflektorspannungen, bei denen das Maximum ($U_0$),
                 der linke Einstiegswert ($U_1$) und der rechte Einstiegswert ($U_2$)
                 am festgelegten Referenzpunkt auf dem Oszilloskop lagen.
                 Für alle drei Moden sind die jeweilige Amplitude und Frequenz angegeben.
                 \label{tab:Moden}}
\end{table}
