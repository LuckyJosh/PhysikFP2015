% !TeX root = Protokoll.tex
Im Folgenden sollen die in der Auswertung erhaltenen Ergebnisse noch einmal abschließend 
zusammengefasst und diskutiert werden, um eine Aussage über deren Plausibilität machen zu können.\\

Aus den Messwerten ergab sich die Feldstärke des Erdmagnetfelds zu
\begin{empheq}{equation}
B_{\mathrm{E}} = \SI{70(10)}{\micro\tesla}.
\end{empheq}
Die relative Abweichung dieses Wertes vom Literaturwert
$\mean{B_{\mathrm{lit}}} = \SI{47}{\micro\tesla} $ \cite{GGU} liegt mit $\Delta_{\mathrm{rel}}B = \num{0.4}$
und ist damit groß, jedoch kann die Größenordnung des Ergebnisses als plausibel bewertet werden.\\
Die Bestimmung des Landé-Faktors hingegen ergab mit 
\begin{empheq}{equation}
g = \num{1.9(2)}
\end{empheq}
einen Wert, bei dem der Literaturwert $g_{\mathrm{lit}} = \num{2.002}$ \cite{NIST} im Bereich des Fehlers liegt.
Die relative Abweichung vom Literaturwert ist mit $\Delta_{\mathrm{rel}}g = \num{0.05} $ auch sehr gering.

Eine mögliche Erklärung für die Abweichungen der Ergebnisses ist eine nicht genaue Justage der Helmholtz-Spulen,
sodass das entsprechende Magnetfeld nicht exakt parallel beziehungsweise antiparallel zum Erdmagnetfeld orientiert war. 
Dieser Fehler führt zu einer konstanten Abweichung der gemessenen Magnetfelder. Eine weitere Fehlerquelle sind 
die Magnetfelder, die in den übrigen Versuche am Versuchsort erzeugt werden. Die Abweichungen aufgrund dieses Fehlers
sind nicht konstant, da die umgebenden Magnetfelder einer zeitlichen Änderung unterliegen. 











