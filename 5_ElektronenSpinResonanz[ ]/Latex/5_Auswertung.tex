% !TeX root = Protokoll.tex
\subsection{Bearbeitung der Resonanzkurven}

Für die Kalibrierung der aufgenommenen Resonanzkurve mussten die minimale $I_{min}$ und 
die maximale Stromstärke $I_{max}$ durch die Helmholtz-Spule notiert werden. Diese Messwerte
sind für die acht aufgenommenen Resonanzkurven in \cref{tab:messwerte_I} eingetragen.
Für die weitere Auswertung wurde der Betrag der Differenz jeweils für beide Orientierungen berechnet.

\FloatBarrier
\begin{table}[!h]
	\centering
	\begin{adjustbox}{width=\textwidth}
	\begin{tabular}{ccccccc}
		\toprule
		HF-Generator Frequenz & Stromstärke & Stromstärke & Stromstärke & Stromstärke & Stromstärke & Stromstärke\\
		$f_e$/\si{\mega\hertz} & $I_{\mathrm{min,+}}$/\si{\milli\ampere} & $I_{\mathrm{max,+}}$/\si{\milli\ampere} & $\Delta I_{+}$/\si{\milli\ampere} & $I_{\mathrm{min,-}}$/\si{\milli\ampere} & $I_{\mathrm{max,-}}$/\si{\milli\ampere} & $\Delta I_{-}$/\si{\milli\ampere}\\
\midrule
		\num{10.602} & \num{61(1)} & \num{464(1)} & \num{403(1)} & \num{-447(1)} & \num{59(1)} & \num{506(1)}\\
		\num{15.912} & \num{46(1)} & \num{679(1)} & \num{633(1)} & \num{-724(1)} & \num{-94(1)} & \num{630(1)}\\
		\num{20.555} & \num{262(1)} & \num{853(1)} & \num{591(1)} & \num{-827(1)} & \num{-361(1)} & \num{466(1)}\\
		\num{24.994} & \num{316(1)} & \num{817(1)} & \num{501(1)} & \num{-906(1)} & \num{-348(1)} & \num{558(1)}\\
		\bottomrule
	\end{tabular}
	\end{adjustbox}
	\caption{Die aufgenommenen Messwerte für HF-Generator Frequenz und die minimale bzw. maximale
Stromstärke an der Helmholtz-Spule für jeweils beide Orientierungen des Magentfelds zum Erdmagentfeld ($+/-$).
Zusätzlich ist noch der Betrag der Differenz der jeweiligen maximalen und minimalen Stromstärke angegeben. \label{tab:messwerte_I}}
\end{table}

\FloatBarrier
  
\FloatBarrier
\begin{figure}[!h]
 \centering
\begin{tikzpicture}			
	\node [draw=white, anchor=south west] (label) at (0,0) {
	\includegraphics[scale=1]{../Grafiken/Resonanzkurve_15912+_cut.pdf}};
	%\draw (0.15,0.15) circle(0.1) ;
	%\draw[step=.5cm,draw=gray] (0.,0.) grid (\textwidth-10,\textwidth - 240);
	\filldraw (0.8,3.55) circle(0.05) node[above] {$x_{\mathrm{A}}$};
	\filldraw (7.5,2.80) circle(0.05)node[above] {$x_{\mathrm{R}}$};
	\filldraw (14.4,3.7) circle(0.05)node[above] {$x_{\mathrm{E}}$};
	\draw[|-|] (0.8,4.5)--(14.4,4.5) node[midway,above] {$d$};
	\draw[|-|] (0.8,2.5)--(7.5,2.5) node[midway,above] {$d_{res}$};
\end{tikzpicture}
 
 \caption{Beispielhafte Darstellung einer Resonanzkurve für $\SI{15.912}{\mega\hertz}$ mit
 	den eingezeichneten Messpunkten und Abständen, die zur Auswertung genommen wurden. \label{fig:resonanzkurve_20555+_cut}}
 \end{figure} 
\FloatBarrier


Zur Bestimmung der Magnetfeldstärke der Helmholtz-Spulen im Resonanzfall wurden die 
aufgenommenen Resonanzkurven vermessen. Dabei wurden in Relation zum Anfangspunkt 
der Kurve bei $I_{min}$ die Distanzen $d$ und $d_{res}$ zum Endpunkt $I_{max}$ und zur 
Resonanzstelle $I_{res}$ vermessen, dargestellt in \cref{fig:resonanzkurve_20555+_cut}.
Diese Messwerte befinden sich in \cref{tab:messwerte_X}.

\FloatBarrier
\begin{table}[!h]
	\centering
	\begin{tabular}{cccc}
		\toprule
		Abstand & Abstand & Resonanzstelle & Resonanzstelle\\
		$d_{+}$/\si{\milli\meter} & $d_{-}$/\si{\milli\meter} & $X_{\mathrm{res,+}}$/\si{\milli\meter} & $X_{\mathrm{res,-}}$/\si{\milli\meter}\\
\midrule
		\num{87(1)} & \num{88(1)} & \num{31(1)} & \num{35(1)}\\
		\num{137(1)} & \num{140(1)} & \num{66(1)} & \num{61(1)}\\
		\num{129(1)} & \num{105(1)} & \num{42(1)} & \num{62(1)}\\
		\num{115(1)} & \num{119(1)} & \num{50(1)} & \num{50(1)}\\
		\bottomrule
	\end{tabular}
	\caption{Die aufgenommenen Messwerte für den Abstand von minimaler und maximaler Spulenstromstärke
sowie die Position der Resonanzstelle auf dem Millimeterpapier des XY-Schreibers. \label{tab:messwerte_X}}
\end{table}

\FloatBarrier




Aus den Abständen zu den rechten Endpunkten der Resonanzkurven und den zuvor berechneten 
Differenzen des maximalen und des minimalen Spulenstroms kann die Änderung des Stroms
mit der Distanz auf dem Millimeterpapier $\tfrac{\Delta I}{d} $ bestimmt werden.
Hieraus kann nun mit 
\begin{empheq}{equation}
	I_{\mathrm{res}} = I_{\mathrm{min}} + \dfrac{\Delta I}{d} \cdot d_{\mathrm{res}}
\end{empheq}
die Stromstärke an der Resonanzstelle bestimmt werden. die entsprechenden Werte sind in 
\cref{tab:messwerte_d} zu finden.

\FloatBarrier
\begin{table}[!h]
	\centering
	\begin{tabular}{cccc}
		\toprule
		Relation & Relation & Resonanzstromstärke & Resonanzstromstärke\\
		$\frac{\Delta I_{+}}{d_{+}}$/\si{\ampere\per\meter} & $\frac{\Delta I_{-}}{d_{-}}$/\si{\ampere\per\meter} & $I_{\mathrm{res,+}}$/\si{\milli\ampere} & $I_{\mathrm{res,-}}$/\si{\milli\ampere}\\
\midrule
		\num{4.63(6)} & \num{5.75(7)} & \num{204(5)} & \num{245(6)}\\
		\num{4.62(4)} & \num{4.50(3)} & \num{350(5)} & \num{449(5)}\\
		\num{4.58(4)} & \num{4.44(4)} & \num{454(5)} & \num{551(5)}\\
		\num{4.36(4)} & \num{4.69(4)} & \num{533(5)} & \num{671(5)}\\
		\bottomrule
	\end{tabular}
	\caption{Werte der Stromänderung mit dem Abstand auf dem Millimeterpapier und die damit 
berechneten Resonanzstromstärken. \label{tab:messwerte_d}}
\end{table}

\FloatBarrier
    
\subsection{Bestimmung der Stärke des Erdmagentfeldes}

Aus den berechneten Resonanzstromstärken lässt sich über \cref{eq:helmholtz} das entsprechende Magnetfeld
an der Resonanzstelle berechnen. Dieses setzt aus dem Magnetfeld der Helmholtz-Spule $B_{\mathrm{H}}$ und
dem Erdmagnetfeld $B_{\mathrm{E}}$ zusammen. Für parallele und antiparallele Ausrichtung dieser Magnetfelder ergeben sich die 
Magnetfeldstärken
\begin{empheq}{align}
	\label{eq:bpara}
	 B_{\mathrm{para}} &= B_{\mathrm{H}} + B_{\mathrm{E}},\\
	 \label{eq:banti}
	 B_{\mathrm{anti}} &= B_{\mathrm{H}} - B_{\mathrm{E}}.
\end{empheq}\\

Durch die Bestimmung des Erdmagnetfelds nach
\begin{empheq}{equation}
\label{eq:erdmagentfeld}
B_{\mathrm{E}} =  \dfrac{B_{\mathrm{para}} - B_{\mathrm{anti}}}{2},
\end{empheq}
lassen sich die Magnetfeldstärken der Helmholtz-Spule bestimmen.


Die aus dem Messwerten berechneten Werte sind in \cref{tab:messwerte_B} eingetragen.
Aus diesen Werten lässt sich aufgrund des Größenverhältnisses 
$B_{\mathrm{H,+}}$ < $B_{\mathrm{H,-}}$ schließen, dass der Index \enquote{-}
die antiparallele und \enquote{+} die parallele Ausrichtung zum Erdmagnetfeld
symbolisiert.


\FloatBarrier
\begin{table}[!h]
	\centering
	\begin{tabular}{ccccc}
		\toprule
		Magnetfeld & Magnetfeld & Magnetfeld & Erdmagnetfeld & Magnetfeld\\
		$B_{\mathrm{H,+}}$/\si{\micro T} & $B_{\mathrm{H,-}}$/\si{\micro T} & $\Delta B_{\mathrm{H}}$/\si{\micro T} & $B_{\mathrm{E}}$/\si{\micro T} & $B_{\mathrm{H}}$/\si{\micro T}\\
\midrule
		\num{286(7)} & \num{344(9)} & \num{57(11)} & \num{28(6)} & \num{315(6)}\\
		\num{492(7)} & \num{630(7)} & \num{138(10)} & \num{69(5)} & \num{561(5)}\\
		\num{637(7)} & \num{774(7)} & \num{136(10)} & \num{68(5)} & \num{705(5)}\\
		\num{748(7)} & \num{941(7)} & \num{193(10)} & \num{96(5)} & \num{845(5)}\\
		\bottomrule
	\end{tabular}
	\caption{Berechnete Stärken des Resonanz-Magnetfelds, die aus diesen gebildete Differenz und die
berechnete Stärke des Erdmangetfeldes. Die von dem Erdmagnetfeld bereinigten Magnetfelder sind zusätzlich angegeben.  \label{tab:messwerte_B}}
\end{table}

\FloatBarrier

Aus den aufgenommenen Messwerten ergibt sich für die Stärke des Erdmagnetfeldes
\begin{empheq}{equation}
%(6.6+/-1.2)e-05
B_{\mathrm{E}} = \SI{70(10)}{\micro\tesla}.
\end{empheq}
Der angegebene Fehler stellt dabei die Standardabweichung vom Mittelwert dar.

\subsection{Bestimmung des Landé-Faktors des Elektrons}

Zur Bestimmung des Landé-Faktors des Elektrons werden die HF-Generator Frequenzen $f_e$ in \cref{fig:magnetfeld} gegen die Magnetfeldstärken
an den Resonanzstellen $B_{\mathrm{H}}$ aufgetragen. Dabei wurden die korrigierten Magnetfeldstärken verwendet aus denen das Erdmagnetfeld
herausgerechnet wurde. 
Diese Darstellung wird verwendet werden, da die beiden Größen $f_e$ und $B_{\mathrm{H}}$ einem linearen Zusammenhang nach
\cref{eq:energie} folgen. Aus dieser Gleichung ergibt sich, dass sich der Landé-Faktor des Elektrons mit der Steigung 
$a = g \dfrac{\mu_{\mathrm{B}}}{h}$ dieser Geraden 
zu
\begin{empheq}{equation}
g = 4\pi\dfrac{m_0}{e}a
\label{eq:lande}
\end{empheq}
ergibt.  


\FloatBarrier
\begin{figure}[!h]
 \centering
 \includegraphics[scale=1]{../Grafiken/Magenetfeld.pdf}
 \caption{Grafische Darstellung der Abhängigkeit der HF-Generator Frequenz von der eingestellten Resonanzmagnetfeldstärke mit lineare Ausgleichsfunktion. \label{fig:magnetfeld}}
 \end{figure} 
\FloatBarrier

Die durchgeführte lineare Regression lieferte die Steigung 
%a [MHz/T]  27170.5569932 +/- 2031.35133291
%b [MHz]    1.52051720938 +/- 1.29581585756
\addtocounter{equation}{-1}

	\begin{empheq}{equation}
	\label{eq:fit_a}
		a = \SI{27(2)}{\giga\hertz\per\tesla}.
	\end{empheq}


%Da die Ausgleichsgerade in \cref{fig:magnetfeld} der Form nach \eqref{eq:energie} entspricht ergibt 
%sich für die bestimmte Steigung
%\begin{empheq}{align*}
%	 a &= g \dfrac{\mu_{\mathrm{B}}}{h}\\ 
%	 &= \dfrac{g}{2} \dfrac{e}{m_0} \dfrac{\hbar}{h}.
%\end{empheq}
%Dadurch erhält man den Betrag des Landé-Faktor des Elektrons durch 
%\begin{empheq}{equation}
%g = 4\pi\dfrac{m_0}{e} \cdot a.
%\end{empheq}
Mit der bestimmten Steigung  \eqref{eq:fit_a} ergibt sich nach \cref{eq:lande} für den Landé-Faktor der Betrag
\begin{empheq}{equation}
g = \num{1.9(2)}.
\end{empheq}

