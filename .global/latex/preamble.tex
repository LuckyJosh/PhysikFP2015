%%%%%%%%%%%%%%%%%%%%%%% Grundeinstellungen %%%%%%%%%%%%%%%%%%%%%%%%%%%
% 'Artikel' Dokumentenklasse und Standardschriftgröße  
\documentclass[12pt]{scrartcl}

% Setzt das Papierformat und den Rand auf 2.5cm                                                      
\usepackage[paper=a4paper,left=2.5cm,right=2.5cm,top=2.5cm,bottom=2.5cm]{geometry}  

% Setzt die Einrückung von Absätzen auf gegebenen Abstand
\setlength{\parindent}{0mm}

% Legt Zeilenabstand fest                                                         
\usepackage[onehalfspacing]{setspace}                                               

% Legt FontKodierung fest
\usepackage[T1]{fontenc}

% Legt Zeichenkodierung fest                                                            
\usepackage[utf8]{inputenc}                                                         

% Neue Deutsche Rechtschreibung
\usepackage[ngerman]{babel}  

% Versieht Referenzen mit Bezeichnung des Objektes                                                      
%\usepackage[ngerman]{varioref}   

% Empfohlener T1-Font für deutsche Texte                                               
\usepackage{lmodern} 

% Deutsche Zitate mit \enqoute{},\enqoute*{} 
\usepackage[babel,german=quotes]{csquotes}

%Einstellungen für Bibliographien mit "biber"                                                              
\usepackage[backend=biber, style=numeric-verb,sorting=none]{biblatex}

%%%%%%%%%%%%%%%%%%%% Seitenlayout %%%%%%%%%%%%%%%%%%%%%%%%%%%%%%%%%%%%

% Ermöglicht detailierte Bearbeitung der Kopf- und Fußzeile
\usepackage{fancyhdr} 

% Setzt Kopf und Fußzeile zurück                                                              
\fancyhf{} 	     
         
% Höhe der Kopfzeile                                                          
\setlength{\headheight}{28.0pt}   

% Höhe der Fußzeile                                                  
\setlength{\footskip}{18.0pt}                                                      

% Dicke des Kopfzeilentrennstrichs
\renewcommand{\headrulewidth}{.5 pt}     

% Dicke des Fußzeilentrennstrichs                                     
\renewcommand{\footrulewidth}{.5 pt}                                                 

%Test ob Variable gesetzt ist oder nicht
%\ifdefined\EN% 
	% Angabe Links-Oben
\lhead{\textbf{\EN}}
%\else%
%    \lhead{\textbf{\textcolor{red}{VERSUCHNAME!!!}}}
%\fi%      
                                                            
% Angabe Mitte-Oben
%\chead{}  

% Angabe Rechts-Oben                                                                         
\rhead{\today}  

% Angabe Links-Unten                                                                    
%\lfoot{}        

% Angabe Mitte-Unten                                                                   
\cfoot{\textbf{\thepage\ von \pageref{LastPage}}}  

% Angabe Rechts-Unten                                 
\rfoot{}                                           

% Anwenden des erweiterten Seitenlayouts
\pagestyle{fancy}      
%%%%%%%%%%%%%%%%%%%%%%% Referenzen %%%%%%%%%%%%%%%%%%%%%%%%%%%%%%%%%%%

% Macht die letzte Seitenzahl referenzierbar mit \pageref{Lastpage}
\usepackage{lastpage}

% Ermöglicht Referenzen im Dokument                                                               
\usepackage{hyperref}
	%Keine Hervorhebung der Referenzen
	\hypersetup{hidelinks}

%Verbesserte Referenzen
\usepackage[german]{cleveref}
                                                              
%%%%%%%%%%%%%%%%%%%%%%%% MINT %%%%%%%%%%%%%%%%%%%%%%%%%%%%%%%%%%%%%%%%
% Fügt mathematische Symbole hinzu, setzt Grenzen, Limiten und Indizes unter das Symbol und nicht dahinter
\usepackage[sumlimits,intlimits,namelimits]{amsmath}  

% Fügt Symbole wie z.B. Zahlenmengen wie $\mathbb{R}$ hinzu                             
\usepackage{amssymb}    
                                                             
% Beweißumgebung
\usepackage{amsthm}  

% Font für Mathematikumgebung                                                              
\usepackage{amsfonts}    
                                                          
% Verbesserte Gleichungsumgebung mit \begin{empheq}[<Aussehen>]{<Umgebungstyp>} ... \end{empheq}
\usepackage{empheq}      

% Ergänzungen für physikalische Arbeiten
\usepackage{physics}

% Chemische Struktur und Summenformeln mit \ce{<Summenformel>}                                                          
%\usepackage[version=3]{mhchem}  

% Chemische Valenzstrichformeln für ganze Moleküle mit\chemfig{<Molekül-Aufbau>}                                                   
%\usepackage{chemfig}                                                               

% Verbesserte Formatierung von größen mit Einheiten 
\usepackage{siunitx}                                                               
	% 'Mal'-Zeichen auf \cdot und Dezimaltrennzeichen auf ',' 
	\sisetup{locale = DE,prefixes-as-symbols = true}                                   
                                                                            
	% Vereinfachtes eintragen von Unsicherheiten mit '42.6(4)' --> '42.6 +/- 0.4'          
	\sisetup{separate-uncertainty=true}                                                                                                                    

 
                                                            
% Fügt verbesserte Vektorpfeile hinzu \vv{<Vektorname>} 
\usepackage[b]{esvect}  

% Brüche mit "/" im Text mit \sfrac{}                                                            
\usepackage{xfrac}

% Darstellung von 2D Feldern 
\usepackage{array}

% Differentialoperatoren,-quotienten und Klammern mit \od[]{}{}, \pd[]{}{}, \del{}, \sbr{}, \cbr{}
\usepackage{commath}

% Pseudocode-Umgebungen 																
\usepackage{algorithmicx}
\usepackage{algpseudocode}

% Einbinden von SourceCode Dateien (listing)
\usepackage{listings}
	% Auswählen der Programmiersprache
	\lstset{language=Python}

%%%%%%%%%%%%%%%%% Seiten- und Floateinstellungen %%%%%%%%%%%%%%%%%%%%%
% Einbinden von Grafiken mit '\includeudegraphics[<Optionen>]{<Grafikpfad>}' und Veränderungen im Text, wie z.B. Schriftfarbe 
\usepackage{graphicx} 

% Fügt Möglichkeit für textumflossende Grafiken und Tabellen hinzu \begin{floating<figure/table>}[option]{width} ... \caption ... \end{floatingfigure}                                                              
\usepackage{floatflt}        
\usepackage{wrapfig}

% Ermöglicht das Hinzufügen von Unterabbildung zu einer Abbildung                                                 
\usepackage{subfig} 

% Verhindert das Wandern von "floating" Umgebungen über eine Bestimmte Grenze (hier: Sections) oder mit \FloatBarrier                                                             
\usepackage[section]{placeins}

% Ermöglicht Zeichnungen im Dokument \begin{tikzpicture} ... \end{tikzpicture}
\usepackage{tikz}  
	% Fügt zusätzlichen Pfeilspitzen hinzu                                                                 
	\usetikzlibrary{arrows}                                                         
	\usetikzlibrary{calc}    
% Besseres Tabellenlayout                                                        
\usepackage{booktabs}

% Skalierbare Umgebung für Tabellen und Bilder
\usepackage{adjustbox}

% Bearbeiten von Bild-/Tabellenunterschriften
\usepackage[font=small,labelfont=bf]{caption}

% Seiten im Querformat mit \begin{landscape}...\end{landscape} 
\usepackage{pdflscape}

% Einbinden von Seiten einer anderen .pdf-Datei
\usepackage{pdfpages}

% Ermöglicht detailierte Einstellungen an Aufzählungssymbolen
\usepackage{enumitem} 

% Ermöglicht das Einfügen mehrerer Einträge in eine Tabellenzelle, getrennt von einem '\'
% \backslashbox{<Eintrag unten-links>}{<Eintrag oben-rechts>} TIPP: Leerzeichen                                                          
%\usepackage{slashbox}   

%Einbinden von Textdateien
\usepackage{fancyvrb}
% redefine \VerbatimInput
\RecustomVerbatimCommand{\VerbatimInput}{VerbatimInput}%
{fontsize=\footnotesize,
	%
	frame=lines,  % top and bottom rule only
	framesep=1em, % separation between frame and text
	rulecolor=\color{gray},
	%
	label=\fbox{\color{black} FILENAME},
	labelposition=topline,
	%
	commandchars=\|\{\}, % escape character and argument delimiters for
	% commands within the verbatim
	commentchar=\#       % comment character
	}
	
%%%%%%%%%%%%%%%%%%%%%%%%%%%%%%%%%%%%%%%%%%%%%%%%%%%%%%%%%%%%%%%%%%%%%%

% Fügt verbesserte Unterschtreichungen hinzu, z.B. doppelt, gezackt, gewellt, etc. mit \uline{},\uuline{},
\usepackage[normalem]{ulem}                                                                                                               

%Verbesserte Verwendung von Daten
\usepackage{scrdate}

% Zusaätzliche Symbole
\usepackage{pifont}

% Fügt extra Symbole hinzu 
\usepackage{textcomp}  
%%%%%%%%%%%%%%%%%%%%%%%%%%%%%%%%%%%%%%%%%%%%%%%%%%%%%%%%%%%%%%%%%%%%%%

\title{} 
\author{} 
% Literaturfile
\addbibresource{../../.global/latex/literatur.bib}
%%%%%%%%%%%%%%%%%%%%Aänderungen & Eigene Befehle%%%%%%%%%%%%%%%%%%%%%%                                                     
% Abstand zwischen Text und Fußnoten
\setlength{\skip\footins}{2cm}  

% Abstand zwischen Fußnoten                                                    
%\setlength{\footnotesep}{2cm}

% Abstand zwischen \items 
%\setlength{\itemsep}{7.5pt}

% Änderung der Fußnotenmarkierungen (hier: Zahlen in Kreisen)																    % Fußnoten mit Zahlen in Kreisen
\renewcommand\thefootnote{\ding{\numexpr171+\value{footnote}}}


\renewcommand{\i}{\ensuremath{\textsl{i}}}
\newcommand{\e}{\ensuremath{\textsl{e}}}
\renewcommand{\Im}{\mathrm{Im}\,}
\renewcommand{\Re}{\mathrm{Re}\,}


% Funktionsmakros mit größenvariablen Klammern benötigt \usepackage{commath}
\newcommand{\E}[1]{\e^{#1}}
\newcommand{\Exp}[1]{\exp\!\del{#1}}
\newcommand{\Ln}[1]{\ln\!\del{#1}}
\newcommand{\Log}[2][10]{\log_{#1}\!\del{#2}}
\newcommand{\Sin}[2][]{\sin^{#1}\!\del{#2}}
\newcommand{\Cos}[2][]{\cos^{#1}\!\del{#2}}
\newcommand{\Tan}[2][]{\tan^{#1}\!\del{#2}}
\newcommand{\Sinh}[1]{\sinh\!\del{#1}}
\newcommand{\Cosh}[1]{\cosh\!\del{#1}}
\newcommand{\Tanh}[1]{\tanh\!\del{#1}}
\newcommand{\Arcsin}[1]{\arcsin\!\del{#1}}
\newcommand{\Arccos}[1]{\arccos\!\del{#1}}
\newcommand{\Arctan}[1]{\arctan\!\del{#1}}
\newcommand{\Arsinh}[1]{\arsinh\!\del{#1}}
\newcommand{\Arcosh}[1]{\arcosh\!\del{#1}}
\newcommand{\Artanh}[1]{\artanh\!\del{#1}}

\newcommand{\Cov}[2]{\text{cov}\!\del{#1,#2}}
\newcommand{\Erw}{\mathrm{E}}


%\DeclarePairedDelimiter{\abs}{\lvert}{\rvert}
\DeclarePairedDelimiter{\mean}{\langle}{\rangle}

%%%%%%%%%%%%%%%%%%%%%%%%%%%%%%%%%%%%%%%%%%%%%%%%%%%%%%%%%%%%%%%%%%%%%%%%%%
\usepackage{etoolbox}

%%%%%%%%%%%%%%%%%%%%%%%%%%%%%%%%%%%%%%%%%%%%%%%%%%%%%%%%%%%%%%%%%%%%%%%%%%


