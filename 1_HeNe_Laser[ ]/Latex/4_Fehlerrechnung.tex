Im folgenden Abschnitt werden die, für die Auswertung der aufgenommenen Daten
verwendeten, Gleichungen aufgezeigt und erläutert.\\
Der Mittelwertes aus mehreren Ergebnissen einer Messung 
wurde mit Hilfe von 
\begin{empheq}{equation*}
	\mean{x} = \sum_{i = 0}^{n}x_i
\end{empheq}
berechnet.
Für die Berechnung der statistischen Abweichung wurde die folgende Gleichung verwendet.
\begin{empheq}{equation*}
\sigma_{x} = \sqrt{\frac{1}{n-1}\sum_{i = 0}^{n}(x_i-\mean{x})^2}
\end{empheq}
Für die Ungenauigkeit der aufgenommenen Messwerte wurde im allgemeinen die kleinste Skaleneinheit des verwendeten Messgeräts
angenommen.
Für die Fehlerfortpflanzung dieser Unsicherheiten wurde die 
gaußsche Fehlerfortpflanzung verwendet.
Damit berechnet sich der Fehler $\sigma_y$ einer Größe $y = y(\va{x})$, mit den Messgrößen $\dim{\va{x}} = n$, wie folgt.
\begin{empheq}{equation*}
\sigma_{y} = \sqrt{\sum_{i = 0}^{n}{\qty(\pdv{f}{x_i}\sigma_{x_i})^2}}
\end{empheq}

Die in der Auswertung angefertigten Regressionskurven wurden mit Hilfe der \emph{Python}-Bibliothek \emph{scipy} \cite{SciPy}
durchgeführt.
 